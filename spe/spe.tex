\section{谱估计子程序}

\subsection{简介}
SPE,全称为Spectrum Estimation Subprocess,在SAC中键入~\verb+spe+~命令即可
进入谱估计子程序。该子程序主要用于处理稳态随机过程,包含了如下三种谱估计方法:

\begin{description}
\item [PDS] 能量密度谱
\item [MLM] 最大似然方法
\item [MEM] 最大熵方法
\end{description}

这三种方法都是间接法,因为它们都用了采样相关函数而不是数据本身来估计谱内容。

\subsection{SPE命令}
SPE子程序中包含了一些专门的命令,同时也可以使用SAC的部分命令。这里只列出SPE专属的命令。
\begin{itemize}
\item \nameref{spe:cor} 计算互相关函数
\item \nameref{spe:mem} 用最大熵方法计算谱估计
\item \nameref{spe:mlm} 用最大似然法计算谱估计
\item \nameref{spe:pds} 用能量密度谱方法计算谱估计
\item \nameref{spe:plotcor} 绘制相关函数
\item \nameref{spe:plotpe} 绘制RMS预测误差函数
\item \nameref{spe:plotspe} 绘制谱估计
\item \nameref{spe:readcor} 读取相关函数
\item \nameref{spe:writecor} 将相关函数以SAC文件格式写入磁盘
\item \nameref{spe:writespe} 将谱估计以SAC文件格式写入磁盘
\end{itemize}

\subsection{理论}
SPE主要用于分析稳态随机过程。它实现了三种不同的间接的谱估计方法。它们
之所以称为是间接的是由于它们不直接从数据出发去做谱估计,而是从由数据
求出的样本相关函数出发去做频谱估计。选择间接方法完全是一种偏爱,因为直接的
频谱估计技术也是可以用的。相关函数本身也是一个有用的函数,在进行频谱估计
的过程中你会了解这一点。SPE的谱估计类型为频率域中的功率密度谱,其频谱被
定义在一定的频率范围内,于是在一些频带中随机过程的功率即为这个频带的功率
谱密度的积分。

\subsection{用户控制}
SPE可以使用户控制频谱估计过程中的一些细节。对那些有频谱估计的经验的人来说,
这是很理想的。对那些不想过细地研究有关理论的用户也提供了便于使用的缺省值。
在测定相关函数时用户可以对数据窗口的类型、尺寸和使用的窗口数进行选择。一般
地讲,这些参数控制了谱分析的分辨率,以及最后的谱估计中的方差。另外,数据的
预白化可以指定为测定相关函数的过程的一部分,预白化对减缓严重的窗口``混淆''
现象是很有用的,``窗口混淆''有可能发生在具有大动态范围的谱估计过程中。
发生在预白化时的频谱失真在最后的结果中进行补偿。在这个过程中,数据的预白化
使用了低阶的预测误差滤波器。

\subsection{算法}
用户可以有三种谱估计算法的选择:功率密度谱、最大似然法和最大熵法。

PDS法相当简单,样本相关函数乘以相关窗,然后对结果进行FFT以获得频谱估计结果。
用户还可以对窗的类型和尺寸进行选择。

MLM法生成一个频谱,这种频谱是一个经过平滑处理的功率密度谱的参量估计。
用户可以选择参量的个数。

MEM估计是另一个参量方法,它使用一个预测误差滤波器对数据进行预白化。这个
谱估计的结果反比于滤波器的功率频率响应。用户可以选择预测误差滤波器的阶数。

\subsection{诊断}
除了频谱之外,一些诊断函数也可以计算并标绘出来。预测误差可以被标绘为阶的
函数。这个图可用来为应用于MEM方法的预测误差滤波器选择一个较好的尺寸。由于
进行PDS估计的算法已经众所周知,所以在SPE中给出了关于这种方法的更多的诊断
信息。90\%置信区间以及估计的频率分辨率可以通过理论进行估算。这些值都可以
在PDS的频谱上显示出来。

\subsection{同主程序的区别}
在SPE和SAC主程序之间有两个主要的区别。SPE一次只能处理一个数据文件,这是因为
SPE在运行期间生成并保存了大量的辅助函数(例如:相关函数、预测误差函数以及
谱估计函数自身)。这种对单个数据文件的限制将在未来的版本中去掉。第二个不同点
是,与SAC不同,SPE中具有自己特有的执行不同指令的次序。

\subsection{初始化}
执行SPE命令时即调用了SPE软件包。调用的同时也定义了各种SPE参数的缺省值。
数据文件在进入SPE之前或进入SPE的任何时间均可以使用READ命令读入,一旦读入
新的文件,系统中将为前面所述的辅助函数生成一个空间。

\subsection{相关}
可以使用\nameref{spe:cor}命令计算相关函数,用\nameref{spe:writecor}命令
可以激昂相关函数作为SAC的数据文件保存起来,还可以用\nameref{spe:readcor}
命令再将它们读回SPE中去,这比每次都重复计算相关函数要更为简单。在数据文件
很长的时候尤为如此。此时用户也可以使用\nameref{spe:plotcor}命令来看一下
相关函数。如果用户准备使用MEM方法的话,还可以使用\nameref{spe:plotpe}命令
来看一下预测误差函数。

\subsection{估计}
用户可以使用\nameref{spe:pds}、\nameref{spe:mlm}、\nameref{spe:mem}命令来
选择三种频谱估计中的任何一种。每一种方法都有自己的选项,你可以使用
\nameref{spe:plotspe}命令来检验谱分析结果。有几种确定比例的选项可以使用。
同样的你也可以使用\nameref{spe:writespe}命令将谱估计的结果作为SAC的数据
文件保存起来。

\subsection{终止}
可以使用quitsub命令终止谱估计子程序,或使用\nameref{cmd:quit}命令终止整个SAC程序的运行。
