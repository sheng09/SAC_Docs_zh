\section{SAC宏中的变量}
\SACTitle{SAC宏中的暂存块变量:}
下面看一看暂存块变量在宏文件中如何使用。在接下来的例子中,第一值是一个变量,其它值是根据
第一个值计算出来的:
\begin{SACCode}
  $KEYS FILES VALUE1
  $DEFAULT VALUE1 4
  READ $FILES
  EVALUATE TO VALUE2 $VALUE1 * 2
  EVALUATE TO VALUE3 %VALUE2 + 1
  MUL $VALUE1 %VALUE2 %VALUE3
  FFT
  BG SGF
  PSP AM 
\end{SACCode}
按照下面的命令执行宏文件:
\begin{SACCode}
SAC> macro MYSTUFF VALUE1 6 FILES ABC DEF XYZ 
\end{SACCode}

\SACTitle{联接:}
你可以在宏参数的前后加上任意的文本、暂存块变量、头段变量。想要在参数前面添加文本的
话就直接在参数前加上需要的字符串即可(这样不会有任何的歧义,因为变量与新增字符之间
是有一个\%或者\$隔开的);若是想在参数之后添加字符串,那么必须在该参数后、字符串前
重复分割符(\$、\%或者\&,请回顾一下这些分割符分别对应什么样的变量)。如下的的例子所示:

假设宏参数STATION值为ABC,那么\$STATION\$.Z的值就是ABC.Z

假设暂存块变量TEMP值为ABC那么XYZ\%TEMP的值就是XYZABC,而\%TEMP\%XYZ的值为ABCXYZ

假设文件Z的头段变量KA值为IPU0那么(\&Z,KA\&)的值就是(IPU0)
