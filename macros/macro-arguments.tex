\section{宏参数}
\subsection{SAC宏参数}
SAC宏不只是一堆命令的集合,SAC宏文件也可以有自己的参数。
上面的例子很简单但是不够灵活,如果你想要每次读取不同的文件或者乘以不同的值那么你必须
每次都修改这个文件,让宏文件在执行之前允许用户输入参数可以大大增加宏文件的灵活性。
下面将对先前的宏文件进行修改以使其可以接收文件名作为参数:
\begin{SACCode}
  r $1 $2 $3
  mul 4 8 9
  fft
  bg sgf
  psp am
\end{SACCode}
这些以\$开头的变量代表宏文件接收的参数,其分别代表第一、二、三个参数,用下面的命令
执行这个宏文件:
\begin{SACCode}
SAC> macro MYSTUFF ABC DEF XYZ 
\end{SACCode}
可以用下面的命令再次执行这个宏文件读取不同的文件:
\begin{SACCode}
SAC> macro MYSTUFF AAA BBB CCC
\end{SACCode}

\SACTitle{关键字驱动参数:}
关键字驱动参数允许用户按照任意顺序输入参数,这也使得宏文件的内容变得简单易懂。
当参数的数目以及宏文件的大小不断增大的时候这就变得更加重要了。
下面将再一次修改这个例子以使其可以接受文件列表以及乘数的列表:
\begin{SACCode}
  $keys files values
  r $files
  mul $values
  fft
  bg sgf
  psp am
\end{SACCode}
这样一个简单的修改增加了宏文件的灵活性和可读性。第一行表明有两个关键字,
一个称为``files'',一个称为``values''。可以按照下面的输入来执行这个宏文件:
\begin{SACCode}
SAC> macro MYSTUFF FILES ABC DEF XYZ VALUES 4 8 9
\end{SACCode}
因为参数的顺序不再重要,所以你可以像下面这样输入:
\begin{SACCode}
SAC> MACRO MYSTUFF VALUES 4 8 9 FILES ABC DEF XYZ
\end{SACCode}
这个宏文件并不限于读取三个文件,它对于文件的数目没有限制,只要文件数与值数目相匹配就好。

\SACTitle{默认参数值:}
有些时候会遇到这样的情况,宏文件的有些参数在多次执行的过程中经常但并不总是拥有相同的值。
为这些参数提供缺省值可以减少输入那些相同值的次数同时又保有宏参数本身的灵活性。如下例所示:
\begin{SACCode}
  $keys files values
  $default values 4 8 9
  r $files
  mul $values
  fft
  bg sgf
  psp am
\end{SACCode}
在宏文件的第二行指定了这些参数的缺省值,如果在执行宏文件时不输入VALUES的参数值那么这些
参数将使用缺省值:
\begin{SACCode}
SAC> macro MYSTUFF FILES ABC DEF XYZ 
\end{SACCode}
如果想要使用不同的值,可以像下面这样输入:
\begin{SACCode}
SAC> macro MYSTUFF VALUES 10 12 3 FILES ABC DEF XYZ
\end{SACCode}

\SACTitle{参数请求:}
如果你在执行宏文件时没有输入参数而这些参数又没有缺省值,SAC 将会在终端中提示你输入相应的
参数。在上面的例子中,如果你忘记输入参数则会出现下面的情况:
\begin{SACCode}
SAC> MACRO MYSTUFF
FILES?
SAC> ABC DEF XYZ 
\end{SACCode}
注意到SAC 并不会提示输入VALUES的值,因为它们已经有了缺省值。SAC并非在一开始就提示输入
参数,其等到需要计算参数值却发现没有缺省值或者输入值时才会提示需要输入该参数。