\SACTitle{Do Loops:}
这个特性允许你重复执行一系列命令。可以通过固定循环次数、遍历元素列表或者设定一定条件
来执行一定命令,你也可以中途中断一次循环。循环的最大嵌套次数为10次。其语法可以有多种形式:
\begin{SACCode}
  DO variable = start, stop, {,increment}
  	commands
  ENDDO

  DO variable FROM start TO stop { BY increment}
  	commands
  ENDDO

  DO variable LIST entrylist
  	commands
  ENDDO

  DO variable WILD {DIR name} entrylist
  	commands
  ENDDO

  WHILE expr
  	commands
  ENDDO
  BREAK
\end{SACCode}
其中大写字符串均为关键字,不可更改:
\begin{itemize}
\renewcommand\labelitemi{\dag}
\item variable是循环变量名,在变量名前加上\$即可以在do循环中使用该变量
\item start、stop、increment 循环变量的初值、终值、增值,start、stop必须为整型数,
		increment缺省值为1
\item entrylist 是do循环执行时变量可以取的所有值的集合,值之间以空格分开,其可以为整型、
	浮点型或字符型。DO WILD中entrylist由字符串和通配符构成,循环执行前,这个列表将根据
	通配符扩展为一系列文件名。
\end{itemize}

\SACTitle{Do循环例子:}
第一个宏文件对文件使用了DIF命令以预白化数据,施加Fourier变换,然后使用DIVOMEGA命令去除
预白化的影响,有时需要在做变换之前多次预白化,那么就可以这样写:
\begin{SACCode}
  $KEYS FILE NPREW
  $DEFAULT NPREW 1
  READ $FILE
  DO J = 1 , $NPREW
  	DIF
  ENDDO
  FFT AMPH
  DO J = 1 , $NPREW
  	DIVOMEGA
  ENDDO
\end{SACCode}
下面这个例子,用相同的数据绘制5个不同的两秒时间窗的质点运动矢量图:
\begin{SACCode}
  READ ABC
  SETBB TIME1 0
  DO TIME2 FROM 2 TO 10 BY 2
  	XLIM %TIME1 $TIME2
  	TITLE 'Particle Motion from %TIME1 to $TIME2$'
  	PLOTPM
  	SETBB TIME1 $TIME2
  ENDDO 
\end{SACCode}
想一想为什么在TITLE命令中为什么TIME2的后面也需要一个\$?
在下面的例子中,一个宏文件调用另一个名为PREVIEW的宏文件,通过do循环以达到多次调用PREVIEW的目的:
\begin{SACCode}
  DO STATION LIST ABC DEF XYZ
  	DO COMPONENT LIST Z N E
  		MACRO PREVIEW $STATION$.$COMPONENT$
 	ENDDO
  ENDDO
\end{SACCode}
在下面的例子中我们修改上一个宏文件使得其可以处理目录MYDIR中所有以".Z"结束的文件:
\begin{SACCode}
  DO FILE WILD DIR MYDIR *.Z
 	MACRO PREVIEW $FILE
  ENDDO 
\end{SACCode}
最后一个例子有三个参数,第一个是文件名,第二个是一个常数,第三个是一个阀值。宏文件读取了
一个数据文件,然后每个数据点乘以一个常数直到其超过某一阀值:
\begin{SACCode}
  READ $1
  WHILE &1,DEPMAX GT $3
  	MUL $2
  ENDDO 
\end{SACCode}
另一个与BREAK有关的宏文件:
\begin{SACCode}
  READ $1
  WHILE 1 GT 0
  	DIV $2
  	IF &1,DEPMAX GT $3
  	BREAK
  	ENDIF
  ENDDO 
\end{SACCode}
这个WHILE循环是一个无限循环,它只能通过BREAK来中断。(这个版本有一个bug,如果初始的最大值
已经低于阀值了,那么BREAK就永远不会被执行)
