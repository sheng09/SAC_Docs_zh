\section{其他特性}
\SACTitle{嵌套与递归:}
SAC提供嵌套功能,不支持递归,但是SAC并不会去检查宏的调用是否保证不是递归,因而需要
用户去保证宏文件不要直接或间接调用自己。

\SACTitle{中断宏:}
有些时候需要临时中断宏文件的执行,用户自己从终端输入一些命令,然后继续执行宏文件。
这个可以利用SAC的pause和resume特性做到。当SAC在宏文件中遇到\$TERMINAL时它会临时停止
执行宏文件,更改提示符为宏名,然后提示从终端输入命令,然后当SAC在终端中看到\$RESUME
时则会停止从终端读取命令继续从宏文件读取。如果你不想再继续执行宏文件中的命令,可以
在终端输入\$KILL,SAC将关闭宏文件,回到上一层。在一个宏文件中可以有多个\$TERMINAL中断。


\SACTitle{从宏文件中执行其他程序:}
你可以在SAC宏内部执行其他程序,可以向程序传递一个可选的执行信息。如果程序是交互式的
你也可以将输入行发送给它,语法如下:
\begin{SACCode}
  $RUN program message
  inputlines
  ENDRUN
\end{SACCode}
宏参数、暂存块变量、头段变量、内联函数均可使用,在程序执行之前它们会被计算,当程序执行
结束,SAC宏会在ENDRUN之后继续执行。


\SACTitle{转义字符:}
有些时候可能会需要在命令中使用\$或者\%而又不想SAC将其解释为一个变量。为了实现这个目的可以
在这些特殊的字符之前加上另一个特殊的字符,称为转义符,SAC中的转义符为@,可以被转义的特殊
符号包括:
\begin{itemize}
\renewcommand\labelitemi{\dag}
\item \$  宏参数扩展字符
\item \%  暂存块变量扩增字符
\item \&  头段变量扩展字符
\item  @  转义字符本身
\item  (  内置函数起始符
\item  )  内置函数结束符
\end{itemize}
