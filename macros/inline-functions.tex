\section{内联函数}

内置函数是SAC实现的一些函数,其可以在SAC命令中使用。在执行命令时,内联函数会首先被
调用,内联函数的结果将替代命令中的内联函数的位置。

SAC提供了如下几类内联函数:
\begin{itemize}
\item 算术运算符;
\item 常规算术运算函数;
\item 字符串操作函数;
\item 其他函数;
\end{itemize}

所有的内联函数的共同形式是:\lstinline{(func)},其中func为内联函数名,在某种程度
上内联函数与前面说到头段变量(\lstinline{& &})和黑板变量(\lstinline)类似,
可以认为是通过\lstinline{( )}引用了内联函数的结果或值。

内联函数支持嵌套,目前最多可以嵌套10层。

\subsection{算术运算符}
算术运算符即常规的加减乘除运算符,但又有不同,其一般形式如下:
\begin{SACCode}
    ( number operator number ) 
\end{SACCode}
所有的操作数都被认为是实型的,所有的算术运算都按照双精度浮点型进行运算;

SAC支持的操作符是包括:\lstinline{ +  -  *  /  ** }

看几个简单的例子:
\begin{SACCode}
SAC> echo on                
SAC> setbb var1 4+7             // 忘记加括号了!"4+7"被当成了字符串
 setbb var1 4+7
SAC> setbb var2 (4+7)           
 setbb var2 (4+7)
 ==>  setbb var2 11             // 4+7=11
SAC> setbb var3 (4+7/3)         // 优先级正确
 setbb var3 (4+7/3)
 ==>  setbb var3 6.33333
SAC> setbb var4 ((4+7)/3)       // 括号改变优先级
 setbb var4 ((4+7)/3)           // 可以看作是内联函数的嵌套
 ==>  setbb var4 3.66667
SAC> setbb var1 ( ( 4 + 7 ) / 3 )   // 支持空格
 setbb var1 ( ( 4 + 7 ) / 3 )
 ==>  setbb var1 3.66667
\end{SACCode}

\subsection{常规算术运算函数}
SAC提供了20个常规算术运算函数,其基本形式为\lstinline{(func arg1 arg2 ...)}。

\begin{table}[H]
\centering
\caption{常规算数运算函数}
\label{table:regular-arithmetic-functions}
\begin{tabular}{l|l|l}
	\toprule
	命令	&	语法	&	功能	\\
	\midrule
	add		&	( add v1 v2 ... vn )	    &	v1+v2+...+vn	\\
	subtract&	( subtract v1 v2 ... vn )   &	v1-v2-...-vn	\\
	multiply&	( multiply v1 v2 ... vn )   &	v1*v2*...*vn	\\
	divide	&	( divide v1 v2 ... vn )	    &	v1/v2/.../vn	\\
	absolute&	( absolute v )			    &	取绝对值	\\
	power	&	( power v )				    &	取10的v次方     \\
	alog10	&	( alog10 v)				    &	以10为底取v的对数	\\
	alog	&	( alog v)				    &	取v的自然对数	\\
	exp	    &	( exp v)				    &	取e的v次方	\\
	sqrt	&	( sqrt v) 				    &	求v的平方根	\\
	pi		&	( pi )					    &	返回pi值	\\
    sine    &	( sine v )				    &	正弦(v为弧度,下同)\\
	cosine	&	( cosine v )				&	余弦	\\
    tangent	&	( tangent v )			    &	正切	\\
	arcsine	&	( arcsine v )			    &	反正弦	\\
	arccosine&	( arccosine v ) 			&	反余弦	\\
	arctangent&	( arctangent v )			&	反正切	\\
    integer &	( integer v )			    &	取整	\\
	maximum	&	( maximum v1 v2 ... vn )	&	求最大值	\\
	minimum	&	( minimum v1 v2 ... vn )	&	求最小值	\\
	\bottomrule
\end{tabular}
\end{table}

演示如下:
\begin{SACCode}
SAC> echo on processed
SAC> setbb var1 (add 1 3 4)         // 1+3+4
 ==>  setbb var1 8
SAC> setbb var2 (subtract 1 3 4)    // 1-3-4
 ==>  setbb var2 -6
SAC> setbb var3 (multiply 1 3 4)    // 1*3*4
 ==>  setbb var3 12
SAC> setbb var4 (divide 1 3 4)      // 1/3/4
 ==>  setbb var4 0.0833333
SAC> setbb var5 ( absolute -5.1 )   // abs(-5.1)
 ==>  setbb var5 5.1
SAC> setbb var6 ( power 5 )         // 10^5
 ==>  setbb var6 100000
SAC> setbb var7 ( alog10 10000 )    // log10(10000)
 ==>  setbb var7 4
SAC> setbb var8 ( alog 10000 )      // ln(10000)
 ==>  setbb var8 9.21034
SAC> setbb var9 ( exp 5 )           // e^5
 ==>  setbb var9 148.413
SAC> setbb var10 ( sqrt 9 )         // sqrt(9)
 ==>  setbb var10 3
SAC> setbb var11 ( pi )             // PI
 ==>  setbb var11 3.14159
 SAC> setbb var12 ( sine (pi/6) )   // sin(30)
 ==>  setbb var12 0.5
SAC> setbb var13 ((arcsine 0.5)*180/(pi))
 ==>  setbb var13 30
SAC> setbb var14 (integer 3.11)
 ==>  setbb var14 3
SAC> setbb var15 (max 3.11 -1.5 5)  // maximum简写为max
 ==>  setbb var15 5
SAC> setbb var16 (min 3.11 -1.5 5)  // minimum简写为min
 ==>  setbb var16 -1.5
\end{SACCode}

为了对一组数据做归一化,首先要找到所有数据中的绝对最大值,如下:
\begin{SACCode}
SAC> r file1 file2 file3 file4
SAC> echo on processed
SAC> setbb vmax (max &1,depmax& &2,depmax& &3,depmax& &4,depmax&)
 ==> setbb vmax 1.87324
SAC> setbb vmin (min &1,depmin& &2,depmin& &3,depmin& &4,depmin&)
 ==> setbb vmin -2.123371
SAC> div ( max (abs %vmax%) (abs %vmin%) )      // 嵌套
 ==>  div 2.123371 
\end{SACCode}
此例可以通过多重嵌套的方式在单个命令中完成,但上面的写法可读性更强。

\subsection{字符串操作函数}
SAC提供了若干个函数用于字符串的处理,如表\ref{tabel:string=operation-functions}所示:

\begin{table}[H]
\centering
\caption{字符串操作函数}
\label{table:string-operation-functions}
\begin{tabular}{lll}
	\toprule
	命令	&	语法(简写形式)	&	功能	\\
	\midrule
	change		&	( cha s1 s2 s3 ) 	&	在s3中用s1代替s2	\\
	substring 	&	( subs n1 n2 s ) 	&	取s中第n1到第n2个字符\\
	delete		&	( del s1 s2 )		&	从s2中删去s1	\\
	concatenate &	( conc s1 s2 ... sn )	&	将多个字符串拼接起来 \\
	before		&	( bef s1 s2)			&	得到s2中位于s1前的部分字符串\\
	after		&	( aft s1 s2 )			&	得到s2中位于s1后的部分字符串\\
	reply		&	( rep s1 )			&	发送信息s1到终端并得到回应	\\
	\bottomrule
\end{tabular}
\end{table}

下面的例子展示了部分函数的用法:
\begin{SACCode}
change substring delete before after的用法
SAC> setbb month (substring 1 3 '&1,kzdate&')
 ==> setbb month mar
SAC> fg seismo
SAC> SETBB MONTH (SUBSTRING 1 3 &1,KZDATE)
==>  SETBB MONTH MAR
SAC> message (substring 1 5 &1,kevnm)
==>  message K8108
setbb VAL "1234567890"
SAC> message (substring 1 5 %VAL)
message (substring 1 5 %VAL)
==>  message 12345
\end{SACCode}

下面的例子利用concatenate函数定义标题,在标题中使用台站和文件名:
\begin{SACCode}
SAC> fg seis
SAC> p
SAC> echo on
SAC> title '(concatenate 'Seismogram of ' &1,kevnm ' ' &1,kstnm )'
 ==> title 'Seismogram of K8108838 CDV'
SAC> p
SAC> FUNCGEN SEISMOGRAM
SAC> ECNO ON
SAC> TITLE '(CONCATENATE 'Seismogram of ' &1,KEVNM ' ' &1,KSTNM )'
old output  ==> TITLE 'Seismogram of K8108838 CDV'
v101.6 output  ==>  TITLE "(CONCATENATE " Seismogram of " K8108838 " " CDV )"
SAC> title "Seismogram of &1,KEVNM &1,KSTNM"
title "Seismogram of &1,KEVNM &1,KSTNM"
==>  title "Seismogram of K8108838 CDV"
SAC> setbb a (CONCATENATE Seismogram of  &1,KEVNM  &1,KSTNM )
==>  setbb a SeismogramofK8108838CDV
SAC> setbb a (CONCATENATE Seismogram' ' of' '' '  &1,KEVNM  &1,KSTNM )
==>  setbb a Seismogram of  K8108838CDV
SAC> > setbb a 'Seismogram of  &1,KEVNM  &1,KSTNM'
==>  setbb a "Seismogram of  K8108838  CDV"
\end{SACCode}

下面的例子使用reply函数实现了交互:
\begin{SACCode}
SAC> fg seis
SAC> rmean; rtr; taper
SAC> setbb low (reply "Enter low freqency limit for bandpass:")
SAC> setbb high (reply "Enter low freqency limit for bandpass:")
SAC> bp c %low% %high%
\end{SACCode}

下面的例子中reply函数包含了一个默认值值:
\begin{SACCode}
SAC> setbb bbday (reply "Enter the day of the week: [Monday]") 
\end{SACCode}
当这个函数执行时,引号中的字符串将出现在屏幕上,提示用户输入。如果用户输入,SAC会将
输入的字符串作为回应值,如果用户只是敲击回车键,SAC则会使用该默认值即MONDAY。

\subsection{其他函数}
这类函数目前只有一个:GETTIME,用于返回数据中首先出现某个值的时间相对于文件开始时间的时间偏移量(s):
\begin{SACCode}
   (GETTIME MAX|MIN [value])
\end{SACCode}
返回内存中第一个文件中数值为value的数据点的时间(以秒为单位),如果没有指定value,
MAX为文件中第一个大于或等于DEPMAX的数据点的时间;MIN为文件中第一个小于或等于DEPMIN
的数据值点的时间。指定value是为了控制正在查找的数据点的数值。

对于所有的文件有一个最大振幅,要找到这些文件中第一个文件中第一次大于该值所对应的时
间偏移量:
\begin{SACCode}
SAC> r FILE1 FILE2 FILE3 FILE4
SAC> setbb maxtime ( gettime max )
 ==> setbb maxtime 41.87
\end{SACCode}

为了找到一个小于或等于123.45的数据点的时间偏移,可以使用如下命令:
\begin{SACCode}
SAC> setbb valuetime ( gettime min 123.45 )
 ==> setbb valuatime 37.9 
\end{SACCode}