\SACCMD{chnhdr}
\label{cmd:chnhdr}

\SACTitle{概要}
改变指定的头段变量值

\SACTitle{语法}
\begin{SACSTX}
C!HN!H!DR! [file n1 n2 ...] field v [field v ...] [ALLT v]
\end{SACSTX}

\SACTitle{输入}
\begin{itemize}
\item file : 可选关键字,后跟数字列表,由于chnhdr只能对内存中的文件头段进行操作,
    因而只需要给出数字即可指定要对内存中的哪些文件进行操作。
\item n1、n2: 文件号,指定对哪些文件进行操作
\item field : SAC头段变量名。注意为了保证数据内部一致性,下面的头段变量值不可用CHNHDR更改:NVHDR, NPTS, NWFID, NORID, 和NEVID
\item ALLT v: 在所有已定义的时间相关头段变量的值上加v秒,同时将参考时刻减去v秒。常用于
    调整数据的参考时间
\item v : 设置field代表的头段变量的值为v。变量和值的类型必须匹配。
\end{itemize}

\SACTitle{说明}
关于头段变量值v的说明:
\begin{itemize}
\item 对于有内部空格的字符串要用单引号括起来。
\item 逻辑字段用TRUE或FALSE,YES或NO也可以接受。
\item 对于相对时间字段(B, E, O, A, F,Tn),v可以是相对参考时间的时间偏移量,
    也可以是下面的形式~\lstinline{GMT v1 v2 v3 v4 v5 v6}~,其中v1,v2,v3,v4,v5,v6是
    GMT年、儒略日、小时、分钟、秒、毫秒,如果v1是两位数,SAC假设其为当前世纪,
    除非那个时间是未来时间,那种情况下SAC假定是上个世纪,最好还是用4位整数表示年。
\item UNDEF: 关键字,使头段为未定义状态
\end{itemize}

这个命令允许你修改指定的一个或多个文件的头段变量值,在未指定文件号的情况下,则对所有内
存中的文件进行操作。要改变磁盘中文件的头段你还需要使用write或writehdrR命令,SAC
会对新值做有效性检查,不过你可以使用LISTHDR自己检查。

头段中用6个变量定义了参考时刻,这是SAC中唯一的绝对时刻,其它时刻都被转换成相对
于参考时刻的相对时间。可以使用``ALLT v''修改参考时刻以及相对时间。
参考时间被减去了v秒,相对时间被加上了v秒,
这保证了数据的绝对时刻不发生改变。为了方便,你可以通过输入绝对时刻
而非相对时间来改变时间偏移变量的值。绝对时刻首先被转换为相对时间,然后再存入头段中。

\SACTitle{例子}
为了定义内存中所有文件的事件经纬度、事件名::
\begin{SACCode}
SAC> ch evla 34.3 evlo -118.5
SAC> ch kevnm 'LA goes under'
\end{SACCode}

为了定义第二、四个文件的事件经纬度、事件名:
\begin{SACCode}
SAC> ch file 2 4 EVLA 34.3 EVLO -118.5
SAC> ch file 2 4 KEVNM 'LA goes under'
\end{SACCode}

设定初动到时为无定义状态:
\begin{SACCode}
SAC> ch A UNDEF
\end{SACCode}

假设你知道事件的GMT起始时间,你想要快速改变头段中所有的时间变量,使得发震时刻是0
即参考时间为发震时刻,并且所有的相对时间根据这个时间去纠正相对值。

首先用GMT选项设置事件起始时间:
\begin{SACCode}
SAC> ch o GMT 1982 123 13 37 10 103
\end{SACCode}
现在使用LISTHDR检查发震时刻o相对于当前参考时间的描述:
\begin{SACCode}
SAC> lh o
 o = 123.103
\end{SACCode}
现在使用ALLT选项从所有的偏移时间中减去这个值,并加到参考时间上,你同时需要改变描述参考时间类型的字段:
\begin{SACCode}
SAC> ch allt -123.103 iztype iO
\end{SACCode}
注意这里的负号意味着从偏移时间中减去这个值。

\SACTitle{头段变量改变}
几乎所以头段变量都可以被改变。

\SACTitle{错误消息}
\begin{itemize}
\item[-]1006: 字符串变量长度太长,注意每个头段都是有字节限制的。
\item[-]1301: 未读入数据文件。
\end{itemize}

\SACTitle{相关命令}
\nameref{cmd:listhdr}、\nameref{cmd:write}、\nameref{cmd:writehdr}
