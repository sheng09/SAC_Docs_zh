\SACCMD{envelope}
\label{cmd:envelope}

\SACTitle{概要}
利用Hilbert变换计算包络函数

\SACTitle{语法}
\begin{SACSTX}
ENVELOPE
\end{SACSTX}

\SACTitle{说明}
该命令用于计算内存中数据的包络函数。

原始信号为$s(t)$,对其做Hilbert变换得到$H(t)$,将这两个信号合并起来构成复信号
\[
    C(t) = s(t) + i*H(t)
\]

复信号不仅可以用``实部-虚部''形式表示,也可以用``振幅-相位''形式表示:

\[
    C(t) = A(t) e^{i\Phi(t)}
\]

其中$A(t)$即为包络函数,其可以进一步表示为
\[
    A(t) = \sqrt{s(t)^2+H(t)^2}
\]

和~\nameref{cmd:hilbert}~一样,数据点数不得少于201,且超长周期的数据需要
在处理之前进行减采样。

\SACTitle{头段变量}
depmin、depmax、depmen

\SACTitle{相关命令}
\nameref{cmd:hilbert}、\nameref{cmd:decimate}
