\SACCMD{bbfk}
\label{cmd:bbfk}

\SACTitle{概要}
利用SAC内存中的所有文件计算宽频频率-波数谱估计

\SACTitle{语法}
 BBFK [Filter] [NOrmalize] [EPS v] [MLM|PDS] [Exp n] [WAvenumber v] [Size m n] [Levels n] [Db] [Title text] [WRite [ON|OFF fname] [Ssq n] [PRINT pname]]

\SACTitle{输入}
\begin{itemize}
\item FILTER: 使用最近一次FILTERDESIGN命令设计的带通滤波器
\item NORMALIZE: 用Capo方法归一化协方差矩阵,如果各信号道的振幅差别比较大,这是一个好方法
\item EPS v: 调整协方差矩阵的分量值,矩阵对角线的项是(1.0+EPS)的整数倍。
\item MLM: 在高分辨率估计中使用最大似然法
\item PDS: 不采用最大似然法的功率谱密度
\item EXP n: 波数谱增加的幂次
\item WAVENUMBER v: 从中采样谱估计的波数目
\item SIZE m n: 极坐标中等值线的尺寸:m是方位角方向上的采样点数;n是在波数方向上的采样点数。m、n必须为偶数,而且其乘积最大限为40000 
\item LEVELS n: 等值线间隔数
\item DB: 以分贝为单位的对数坐标图形
\item TITLE text: 图形标题
\item WRITE ON|OFF fname: 是否计算二维等值线数据并写入磁盘(xyz类型的SAC文件)。fname是要写入的文件名或路径名。如果没有指定文件名,则默认为BBFK
\item SSQ n: 二维图的尺寸(取沿着正方形每个边的采样数据点),最大允许值为200。
\item PRINT pname: 将结果由打印机打印
\end{itemize}

\SACTitle{缺省值}
BBFK EPS .01 PDS EXP 1 WVENUMBER 1.0 SIZE 90 32 LEVELS 11 WRITE OFF SSQ 100

\SACTitle{说明}
BBFK命令允许用户计算宽频频率-波数谱。

\SACTitle{头段数据}
下面的分情况决定头段的信息:

Case1: 如果参考台站设置在KUSER1中并且其对于所有文件是相同的,则所有文件的USER7和USER8都需要设置为偏移量

Case2: 如果所有文件台站纬度(STLA)以及台站经度(STLO)都设置了,则偏移量通过这些经纬度计算,以第一个文件作为参考台站

Case3: 如果所有文件的USER7和USER8都设置了,则它们直接作为偏移量

Case4: 如果所有文件的事件纬度(EVLA)以及事件经度(EVLO)都设置了,则他们用于计算偏移量,使用第一个台站作为参考台站

\SACTitle{输出}
极性的输出立即被绘制出(不保留),二维输出写入到硬盘。FK的峰值、反方位角以及波数将分别写入暂存块变量BBFK\_AMP, BBFK\_BAZIM 以及BBFK\_WVNBR

\SACTitle{错误消息}
\begin{itemize}
\item[-]尺寸m或者n不是一个偶数。
\item[-]偏移量X、Y、Z未设置在头段变量USER7,8,9中。
\item[-]未找到FILTERDESIGN得到的系数数据,或者滤波器类型不是``BP''
\end{itemize}

\SACTitle{限制}
\begin{itemize}
\item 台站最多允许有100个。
\item 极性等值线的最大尺寸是m x n = 40000.
\item 二维等值线输出的最大尺寸是i = 200.
\end{itemize}

\SACTitle{参考文献}
Nawab, SH, FU Dowla, and RT Lacoss, Direction determination of wideband signals,IEEE Trans. Acous. Speech Sig. Proc., 33: (5), 1114-1122, 1985

\SACTitle{相关命令}
ARRAYMAP

\SACTitle{最近修订}
July 22, 1991 (Version 10.6c)
