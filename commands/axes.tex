\section{axes}
\label{cmd:axes}

\SACTitle{概要}
控制注释轴的位置(注释轴指轴上是否显示刻度数字)

\SACTitle{语法}
AXES ON|OFF|ONLY All|Top|Bottom|Right|Left

\SACTitle{其他形式}
命令AXES的另一种形式是AXIS

\SACTitle{输入}
\begin{itemize}
\item ON: 显示列表列出的注释轴,其他不变
\item OFF: 不显示列表列出的注释轴,其他不变
\item ONLY: 只显示列表列出的注释轴,其他的不显示
\item All: 所有的四个注释轴
\item Top: 绘图上部的X注释轴
\item Bottom: 绘图下部的X注释轴
\item Right: 绘图右部的Y注释轴
\item Left: 绘图左部的Y注释轴
\end{itemize}

\SACTitle{缺省值}
AXES ONLY BOTTOM LEFT	即只显示底部和左边的注释轴

\SACTitle{说明}
坐标轴可以绘制在一张图四边的任意一或多个边,有很多命令可以控制坐标轴长什么样。坐标轴的注释间隔用XDIV命令设定(即隔多长显示一个数字),刻度标记的间距可以用TICKS命令单独控制。

only仅显示后面列表中给出的注释轴,而on和off则仅对列表中的注释轴起到打开
或关闭的作用,对其他不在列表中的注释轴则不起作用。要获得自己想要的效果,
使用on或者off时你必须要知道当前已经显示的轴有哪些,哪些是你想要打开或
关闭的。这是一个有点容易弄错的问题,不如只使用only加上想要显示的轴更加简单一点。

\SACTitle{例子}
\begin{SACCode}
SAC> fg seis
SAC> p           //看看SAC的默认设置,左边和底部有注释
SAC> axes on t   //打开顶部注释,左边和底部注释依然保留
SAC> p           //看到的结果是只有顶部注释,没有左边和底部注释,
                 //这里和说明中强调的不一样,应该是程序的bug,
                 //将on认为是only的简写了
SAC> axes on a   //打开所有注释轴
SAC> axes off b  //仅关闭底部注释轴(off选项和说明是一致的)
SAC> axes only b //仅显示底部注释轴
\end{SACCode}

\SACTitle{相关命令}
XDIV,TICKS

\SACTitle{最近修订}
January 8, 1983 (Version 8.0)

