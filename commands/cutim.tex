\section{cutim}
\label{cmd:cutim}

\SACTitle{概要}
裁剪内存中的文件,其可以从每个文件裁剪多段

\SACTitle{语法}
CUTIM pdw [pdw ... ]

\SACTitle{输入}
\begin{itemize}
\item pdw : Partial Data Window,它包含了一个自变量的起始值和结束值,其定义了文件中要读取的部分。pdw的一般形式为ref offset ref offset
\item ref : 参考值,可以取Z|B|E|O|A|F|Tn(n=0,1...9)中的一个,具体参见cut
\item offset : 加到参考值上的一个正数或负数 
\end{itemize}

\SACTitle{缺省值}
如果起始或结束offset省略则认为其为0,如果起始参考值省略则认为其为Z,如果结束参考值省略则认为其值与起始参考值相同。

\SACTitle{说明}
CUT命令只能要截取的数据点然后等待下一个READ命令,CUTIM则在这个命令给出的时候执行截取操作,
用户可以READ一个文件,然后输入CUTIM以及需要的截取区间,SAC将不通过磁盘直接裁剪文件。CUTIM也允许多对数据截取区间,
用户可以READ三个文件到SAC,然后使用有4个截取区间的CUTIM命令,最终内存中将得到12个文件。

**注意**  N选项对于CUTIM不可用

关于pdw的更多命令参见cut命令

\SACTitle{例子}
如果一个结束时间是用震相拾取信息标记的,并且下一个起始时间是一个相对0时刻的偏移量,
那就要避免给出一个以0时刻为参考的结束时间

T1 T2 0 1000 1500

否则,1000会被当作相对T2的偏移,SAC将认为第二个起始时间(1500)缺少了一个	结束时间

\SACTitle{错误消息}
\begin{itemize}
\item[-]1322: 未定义文件剪裁的起始
	\begin{itemize}
	\item[-]头段中的参考值未定义
	\item[-]这个错误可以用CUTERR命令控制
	\item[-]当这个错误被关闭时,使用磁盘文件起始值作为替代
	\end{itemize}
\item[-]1323: 未定义文件裁剪的结束值
\item[-]1324: 起始裁剪值小于文件开始值
	\begin{itemize}
	\item[-]这个错误可以使用CUTERR来控制
	\item[-]当这个错误被关闭时,使用文件的起始值代替错误值或者在文件起始补0
	\end{itemize}
\item[-]1325: 结束裁剪值大于文件结束值
\item[-]1326: 起始裁剪值大于文件结束值
	\begin{itemize}
	\item[-]错误的cut参数
	\item[-]这个错误不可关闭
	\end{itemize}
\end{itemize}

**注意** 上面的一些错误也可以通过CUTERR命令变成警告

\SACTitle{限制}
目前不支持截取非等间隔数据或谱文件

\SACTitle{相关命令}
CUT, READ, APK, PLOTPK, SYNCHRONIZE, CUTERR

\SACTitle{最近修订}
June 18, 1999 (Version 0.58)
