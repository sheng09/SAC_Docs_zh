\SACCMD{transfer}
\label{cmd:transfer}

\SACTitle{概要}
反卷积以去除仪器响应并卷积以加入其它仪器响应。

\SACTitle{语法}
TRANSfer [ FROM type [options] ] [ TO type [options] ] [ FREQlimits f1 f2 f3 f4 ] [ PREWhitening ON | OFF | n ]

\SACTitle{输入}
\begin{itemize}
\item FROM type : 通过谱域除法做反卷积要去除的仪器类型。 
\item TO type : 通过谱域乘法做卷积来加入的仪器类型。 
\end{itemize}

\SACTitle{缺省值}
TRANS FROM NONE TO NONE

\SACTitle{说明}
在TRANSFTER中默认仪器的输入输出都是位移,在SAC中定义为NONE。因而如果FROM类型或者TO类型未指定,SAC会假定其为NONE。

若输出仪器是NONE,则SAC头段中的IDEP将设置为DISPLACEMENT(单位为nm). 对于下表列出的全部地震仪,TO NONE的输出单位为nm。如果TRANSFER选择为TO VEL或TO ACC,SAC中的头段变量IDEP将根据内存中的波形而改变。

如果TO指定为除NONE、VEL、ACC外的其他类型,内存中的波形将被转换为相应的仪器类型。如果FROM仪器类型是NONE,则不会去除仪器响应,原始的地震道数据认定为位移,这常用于向合成地震图中加入仪器响应。

在同一个SAC会话中,当第二次调用TRANSFER时一定要特别小心,因为第二次调用TRANSFER时会使用第一次调用时的参数,除非显式指定其参数。

很多仪器都有参数以进一步指定其仪器响应,最常见的选项是仪器子类型。极少数仪器需要一些给出数值参数而非子类型选项。详情参见下表。

20年前TRANSFER命令引入时,数据获取系统相对简单。当今,多数数据是从IRIS的数据管理中心下载得到,一般为SEED格式。RDSEED程序可以读取SEED数据,EVALRESP程序可以从RDSEED产生的响应文件RESP中计算出完整的系统响应。EVALRESP程序目前已经嵌入到TRANSFER命令的仪器类型选择中,使得用户可以根据RESP文件对数据做卷积或反卷积来处理仪器响应。

SEED文件的格式说明及程序RDSEED和EVALRESP的代码可以由此获得:\\
\href{http://www.iris.edu/software/downloads/seed_tools/}{http://www.iris.edu/software/downloads/seed\_tools/}

SAC中Laplace/Fourier变换中的符号定义与RDSEED和EVALRESP相同:因果响应的相位随着频率增大而减小。对于位移,SAC的单位是nm,而RDSEED的RESP文件使用m为单位。TRANSFER的EVALRESP选项对此做了纠正以使得输出符合SAC的规定。对于其他选项(FAP, PZ),可能会需要修改其单位。(参见下面的例子)

FREQLIMITS f1 f2 f3 f4 : 所有的地震仪在0频率处都具有零响应。当做反卷积但不卷积其他响应的时候(e.g. ``TO NONE''),就有必要修改超低频率处的响应以防止频率域除0。在高频区域,信噪比通常很低,因而有必要对其响应进行抑制。FREQLIMITS即可满足此要求。FREQLIMITS包含了低通和高通的尖灭器(taper)。四个频率满足f1<f2<f3<f4,尖灭器在f2和f3间为1,在f1以下和f4以上为0。频率f1和f2确定了高通滤波器的特性,频率f3和f4指定了低通滤波器的特性。

f1与f2之间以及f3与f4之间分别为余弦波的四分之一个周期。

如果你想要一个低通滤波器但在低频处不滤波,一种方法是设置f1=-2和f2=-1;如果你想要一个高通滤波器但在高频处不滤波,对于Nyquist频率为0.5,设置f3=10和f4=20。

注意:由于滤波器还有零相位,因而其是非因果。如果数据点数npts不为2的指数幂次,会导致在频段(f1,f4)之外振幅不为0。如果想要数据点数为2的幂次方个,可以参考SAC中的CUT命令。

缺省值中是没有FREQLIMITS选项的,强烈建议在做反卷积时包含FREQLIMITS选项。
\begin{itemize}
\item PREWHITENING: 预白化可以将输入时间序列在变换到频率域之前,进行谱的平化。这	会减小谱值的动态范围,并提高数据在高频的计算精度。默认prewhitening是关闭的。参见WHITEN命令。
\item PREWHITENING ON : 在谱操作之前在频率域进行谱白化,并在谱操作后在时间域做谱白化的补偿。
\item PREWHITENING OFF : 关闭预白化。
\item PREWHITENING n : 打开预白化选项并设置其阶数。如果打开预白化但不指定阶数,则缺省n=6,除非在WHITEN命令中已经修改了预白化阶数。
\end{itemize}


\SACTitle{说明}
如EVALRESP和POLEZERO选项,应该加上FREQLIMITS选项以处理超高频和超低频。


\SACTitle{例子}
为了将NYKM.Z去除台站RSTN的仪器响应,并应用DSS的仪器响应:
\begin{SACCode}
SAC> READ NYKM.Z
SAC> TRANS FROM RSTN SUBTYPE NYKM.Z TO DSS PREW OFF
\end{SACCode}  	

为了去除LLL宽频带仪器响应,应用SRO仪器响应,并对频带做尖灭及预白化:
\begin{SACCode}
SAC> READ ABC.Z
SAC> TRANS FROM LLL TO SRO FREQ .02 .05 1. 2. PREW 2
\end{SACCode}
得到的地震数据的通带,在0.05Hz到1Hz是平的,在0.02Hz以下及2Hz以上为0。

在反卷积前在时间域做二阶预白化,在卷积之后该效应被移除。

为了将电磁仪器响应转换为位移:
\begin{SACCode}
SAC>  READ XYZ.Z
SAC>  TRANSFER FROM ELMAG FREEP 15. MAG 750. TO NONE
\end{SACCode}

\SACTitle{致谢}
Roger Hanscom 首先实现了Keith Nakanishi 的TRANSFER程序的转换。

George 	Randall 加入了预白化选项。

Doug Dodge加入了EVALRESP选项.

\SACTitle{最近修订}
August 2011 (Version 101.5) 
