\SACCMD{setbb}
\label{cmd:setbb}

\SACTitle{概要}
设置暂存块变量值

\SACTitle{语法}
\begin{SACSTX}
SETBB variable  [APPEND] value [variable [APPEND] value ...]
\end{SACSTX}

\SACTitle{输入}
\begin{itemize}
\item variable : 暂存块变量名,可以是一个新变量或一个已经有值的变量,变量名最多32字符长
\item value : 暂存块变量的新值,如果也空格必须用引号括起来 
\item APPEND : 将值加到变量的旧值之后,如果该选项忽略,则新值将代替旧值 
\end{itemize}

\SACTitle{说明}
setbb命令可以给黑板变量赋值,这些值可以通过GETBB命令获取,或在命令中直接引用。
你可以使用EVALUATE对黑板变量做基本算术操作,并将结果保存在新的黑板变量中,你也可以
通过UNSETBB命令删除一个黑板变量。

\SACTitle{例子}
同时设置多个黑板变量:
\begin{SACCode}
SAC> setbb c1 2.45 c2 4.94
\end{SACCode}

稍后在命令中使用这些变量:
\begin{SACCode}
SAC> bandpass corners %c1% %c2%
\end{SACCode}

设置包含空格的暂存块变量:
\begin{SACCode}
SAC> setbb mytitle 'sample filter response'
\end{SACCode}

检查以确保值正确:
\begin{SACCode}
SAC> getbb mytitle
 MYTITLE = Sample filter response
\end{SACCode}

\SACTitle{相关命令}
\nameref{cmd:getbb}、\nameref{cmd:evaluate}、\nameref{cmd:unsetbb}
