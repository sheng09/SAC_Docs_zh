\section{line}
\label{cmd:line}

\SACTitle{概要}
控制绘图中线型的选择

\SACTitle{语法}
LINE [ ON | OFF | Solid | Dotted | n ] [ Increment [ ON | OFF ] ] [ List STANDARD | nlist ]

\SACTitle{输入}
\begin{itemize}
\item ON : 打开线型选项,不改变线型 
\item OFF : 关闭线型开关 
\item SOLID : 改变线型为实线型,并打开线型开关 
\item DOTTED : 改变线型为虚线型,并打开线型开关 
\item n : 将线型设置为n并打开线型开关。线型为0代表关闭线型开关。线型的编号随设备变化 
\item INCREMENT {ON} : 每个数据被绘出之后,按照线型表中的次序改变为另一个线型 
\item INCREMENT OFF : 不改变线型 
\item LIST nlist : 改变线型列表的目录,输入线型编号的列表 
\item LIST STANDARD : 设置为标准线型列表 
\end{itemize}

\SACTitle{缺省值}
LINE SOLID INCREMENT OFF LIST STANDARD

\SACTitle{说明}
这个命令控制绘图时的线型,图形的框架(轴、标题等等)通常使用实线。网格的线型用GRID命令控制。并非所有的图形设备都有除实线型之外的其他线型的,在那些设备上显然这个命令没有什么效果。而对于不同的设备线型号n也可能是不同的。

还有其他命令可以控制数据显示的其他方面。SYMBOL命令用于将每个数据点的值用一个符号显示在图上。

COLOR命令控制彩色图形设备的颜色选择。所有的这些属性都是独立的。如果你想的话你可以选择在每个数据点上选择带符号的蓝色虚线。线型为0代表关闭画线选项。在LIST选项和SYMBOL命令中可以利用线型为0的特性,在同一张图上将某些数据用线显示某些数据用符号显示

\SACTitle{例子}
选择依次变化的线型,从线型1开始:
\begin{SACCode}
SAC> LINE 1 INCREMENT
\end{SACCode}

改变线型表使之包含线型3, 5 和 1:
\begin{SACCode}
SAC> LINE LIST 3 5 1
\end{SACCode}

使用PLOT2在同一个图形上绘制三个文件,第一个使用实线无符号,第二个没有线条,用三角符号,第三个无线条,用十字符号:
\begin{SACCode}
SAC> READ FILE1 FILE2 FILE3
SAC> LINE LIST 1 0 0 INCREMENT
SAC> SYMBOL LIST 0 3 7 INCREMENT
SAC> PLOT2
\end{SACCode}

\SACTitle{相关命令}
SYMBOL, COLOR

\SACTitle{最近修订}
October 11, 1984 (Version 9.1)
