\SACCMD{ifft}
\label{cmd:ifft}

\SACTitle{概要}
作离散反Fourier变换

\SACTitle{语法}
\begin{SACSTX}
IFFT
\end{SACSTX}

\SACTitle{说明}
数据文件必须在之前使用过FFT命令,其操作对象为两种格式的谱文件

\SACTitle{头段变量改变}
B, DELTA和NPTS被改成起始频率,采样频率和变换的数据点数。B, DELTA和NPTS的原始值保存在SB, SDELTA,NSNPTS中,当进行IFFT时这些值回到原来的位置

\SACTitle{错误消息}
\begin{itemize}
\item[-]1301: 未读入文件
\item[-]1305: 对时间序列的非法操作
\item[-]1606: 超过允许作IFFT的最大数据点数
\end{itemize}

\SACTitle{限制}
允许作IFFT的最大数据点数为65536

\SACTitle{相关命令}
\nameref{cmd:fft}
