\SACCMD{evaluate}
\label{cmd:evaluate}

\SACTitle{概要}
计算简单算术表达式

\SACTitle{语法}
EVALUATE [TO TERM|name] [v] op v [op v ...]

其中op为下面中的一个:

ADD|SUBTRACT|MULTIPLY|DIVIDE|POWER|SQRT|EXP|ALOG|ALOG10|

SIN|ASIN|COS|ACOS|TAN|ATAN|EQ|NE|LE|GE|LT|GT

\SACTitle{输入}
\begin{itemize}
\item TO TERM : 结果写入用户终端
\item TO name : 结果写入暂存块变量name.
\item v : 浮点数或整数,SAC中所有的运算都是浮点运算,整数会首先转换为浮点型
\item op : 算术或逻辑操作符
\end{itemize}

\SACTitle{其他形式}
\begin{itemize}
\item + 代替ADD
\item - 代替SUBTRACT
\item * 代替MULTIPLY
\item / 代替DIVIDE
\item ** 代替POWER
\end{itemize}
注意要在每个操作符两边加上空格


\SACTitle{缺省值}
EVALUATE TO TERM 1. * 1.

\SACTitle{说明}
这个命令允许你计算算术或逻辑表达式。算术表达式可以是包含多个操作符的复合表达式,在这种情况下表达式由左向右计算,不支持嵌套功能。逻辑表达式可以只包含一个操作符。计算结果可以写入用户终端或者指定的暂存块变量。暂存块变量可以在稍后的命令中直接使用。这在宏文件中非常有用。

\SACTitle{例子}
下面将一个以度为单位的角度转换为弧度并计算其正切值:
\begin{SACCode}
SAC> EVALUATE 45 * 3.14159 / 180
 0.78540
SAC> EVALUATE TAN 0.78540
 0.9982
\end{SACCode}
注意当省略第一个变量时,SAC假定其值为1。

为什么下面的例子得到了不同的结果? :
\begin{SACCode}
SAC> EVALUATE TAN 45 * 3.14159 / 180
 0.02827
\end{SACCode}
最后回顾一下以前的一个例子,但这里使用暂存块变量:
\begin{SACCode}
SAC> EVALUATE TO TEMP1 45 * 3.14159 / 180
SAC> EVALUATE TAN %TEMP1
 0.9982
\end{SACCode}

\SACTitle{限制}
单个表达式中操作数的最大数目为10.

\SACTitle{相关命令}
GETBB

\SACTitle{最近修订}
April 15, 1987 (Version 10.1)
