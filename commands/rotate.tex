\section{rotate}
\label{cmd:rotate}

\SACTitle{概要}
以一个给定的角度选择数据的两个分量

\SACTitle{语法}
ROTATE [ TO GCP | TO v | THROUGH v ] [ NORMAL | REVERSED ]

\SACTitle{输入}
\begin{itemize}
\item TO GCP : 旋转到大圆弧路径(``great circle path'')。两个分量必须都是水平分量。必须定义台站和事件坐标的头段变量 
\item TO v : 旋转角度为v度,两个分量必须都是水平分量 
\item THROUGH v : 旋转角度为v度,一个分量可以是垂直的 
\item NORMAL : 输出分量为正极性 
\item REVERSED : 输出分量为负极性 
\end{itemize}

\SACTitle{缺省值}
ROTATE TO GCP NORMAL

\SACTitle{说明}
这个命令可以将两个分量以一定角度旋转。每对数据分量必须有相同的台站名、事件名、采样率。在THROUGH选项中两个分量可简单地以给定的旋转角度旋转数据的两个分量,其中一个可以是垂直分量,在水平面上的旋转由北起算顺时针旋转。垂直分量的旋转为由上起顺时针旋转。

如果使用TO选项,两个分量必须都是水平分量,这意味着CMPAZ必须定义CMPINC必须是90度。旋转完成后,每一对的第一个分量将指向TO选项给出的角度,如使用TO GCP选项,则这个分量将指向台站-事件向后方位角加减180度的方向。因此,这个分量由事件指向台站,台站和事件坐标头段变量(STLA,STLO,EVLA,EVLO)必须定义,以便由这些值出发计算向后方位角NORMAL和REVERSED选项也可以仅用于水平旋转。如果使用NORMAL选项,则第二个分量比第一个分量超前90度,如果使用REVERSED选项则第二个分量比第一个分量落后90度。

\SACTitle{例子}
以123.43度旋转一对水平分量:
\begin{SACCode}
SAC> READ XYZ.N XYZ.E
SAC> ROTATE TO 123.43
\end{SACCode}
以大圆弧路径旋转两个水平分量数据集:
\begin{SACCode}
SAC> READ ABC.N ABC.E DEF.N DEF.E
SAC> ROTATE TO GCP
SAC> W ABC.R ABC.T DEF.R DEF.T
\end{SACCode}
上面的例子中如果BAZ头段变量为33度,则径向分量指向213度,切向分量指向303度,如果设置反极性,切向分量指向123度。

\SACTitle{头段变量改变}
CMPAZ, CMPINC

\SACTitle{错误消息}
\begin{itemize}
\item[-]1301: 未读入数据文件
\item[-]2001: 命令需要一个偶数个数据文件
\item[-]2004: 旋转没有足够的头段信息
	\begin{itemize}
	\item[-]对于GCP选项STLA, STLO, EVLA, EVLO必须定义
	\end{itemize}
\item[-]2002: 下面的文件不是一个正交对:
\item[-]2003: 下面的文件不都是水平分量,TO选项只对水平分量作用
\end{itemize}

\SACTitle{最近修订}
January 8, 1983 (Version 8.0)
