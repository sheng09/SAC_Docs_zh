\SACCMD{writehdr}
\label{cmd:writehdr}

\SACTitle{概要}
用内存中的头段值覆盖磁盘文件中的相应头段值

\SACTitle{语法}
W!RITE!H!DR!

\SACTitle{输入}

\SACTitle{说明}
该命令并不会覆盖磁盘文件中的数据,使用WRITE将覆盖头段和数据。当CUT选项打开时,WRITEHDR命令无法使用,内存中的头段被修改以反应CUT选项的效果,但是磁盘上的数据不会被修改。对被CUT的数据使用WRITEHDR命令将可能导致磁盘中的数据产生类似于平移或截断的影响。

\SACTitle{错误消息}
\begin{itemize}
\item[-]1301: 未读入数据文件
\end{itemize}

\SACTitle{头段变量}
更新磁盘的头段

\SACTitle{限制}
参见cut和writehdr

\SACTitle{相关命令}
\nameref{cmd:cut}、\nameref{cmd:write}
