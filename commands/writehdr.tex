\SACCMD{writehdr}
\label{cmd:writehdr}

\SACTitle{概要}
用内存中文件的头段区覆盖磁盘文字中的头段区

\SACTitle{语法}
\begin{SACSTX}
W!RITE!H!DR!
\end{SACSTX}

\SACTitle{说明}
write命令的over选项可以用内存中头段区和数据区覆盖磁盘文件中的头段区和数据区。
该命令用内存中头段区覆盖磁盘文件中的头段区,数据区不会被覆盖。如果使用了cut命令,
读取数据时将仅读入部分数据,内存中的头段区将会做相应修改以反映cut命令的效果,
但是磁盘中的数据并没有被修改,因而此时不能使用writehdr命令。对被cut的数据
使用WRITEHDR命令将可能导致磁盘中的数据产生类似于平移或截断的效果。

\SACTitle{错误消息}
\begin{itemize}
\item[-]1301: 未读入数据文件
\end{itemize}

\SACTitle{相关命令}
\nameref{cmd:cut}、\nameref{cmd:write}
