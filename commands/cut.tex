\SACCMD{cut}
\label{cmd:cut}

\SACTitle{概要}
定义要读入文件中的哪部分

\SACTitle{语法}
CUT [ON|OFF|pdw|SIGNAL]

\SACTitle{输入}
\begin{itemize}
\item ON : 打开cut开关但不改变pdw 
\item OFF : 关闭cut开关
\item pdw : 打开cut选项并且改变pdw。pdw是partial data window。它包含了自变量的一个开始值和一个结束值,这两个值定义了文件中要被读取的部分,其一般格式为ref offset ref offset
	\begin{itemize}
	\item ref : 参考时间,可以取Z|B|E|O|A|F|N|Tn,其中n=0,1...9. 
	\item offset : 相对于参考时间的时间偏移量
	\end{itemize}
\item SIGNAL : 即pdw为 A -1 F +1,代表读取文件中从到时前一秒到信号结束后一秒的数据 
\end{itemize}

\SACTitle{缺省值}
CUT OFF

\SACTitle{说明}
如果起始或结束offset省略则认为其为0,如果起始参考值省略则认为其为Z,如果结束
参考值省略则认为其值与起始参考值相同。若cut选项为关,则读取整个文件,cut为开,
则值读取由pdw定义的部分。

下面的一些助记符号用于代表自变量的确定值:
\begin{itemize}
\item B: 磁盘文件起始值
\item E: 磁盘文件结束值
\item Z: 参考时间
\item N: 将offset解释为数据点数而非自变量。其只用于结束值。
\end{itemize}
下面的助记符号代表储存在SAC头段中的参考时间,常用于cut时间序列文件:
\begin{itemize}
\item O: 事件开始时间
\item A: 初动到时
\item F: 信号结束时间
\item Tn: 用户自定义时间标记,n = 0,1...9
\end{itemize}

上面的助记符再加上可选的offset即定义了一个一个起始或结束值。O、A、 F以及Tn可以使用CHNHDR命令来定义。
A和F也可以用自动事件拾取算法APK拾取得到或者人工绘图拾取(PLOTPK)到。如果你想对于一组有不同参考时间的
文件使用同样的时间窗,你必须在执行cut和read命令之前先使用SYNCHRONIZE命令使得所有文件具有的参考时间。
SYNCHRONIZE修改了文件的头使得所有文件具有相同的参考时间,并且调整所有相对到时,使得事件的GMT时间保持
不变。由于cut命令作用在磁盘文件的头段,你必须在SYNCHRONIZE命令之后、READ命令之前,使用WRITEHDR将修改
后的头段写入磁盘文件中,这样才能得到正确的结果。

\SACTitle{例子}
在下面的例子展示了pdw的可能情况,我们假定时间是自变量,单位为秒:
\begin{SACCode}
 B E  磁盘文件开始到文件结束-与cut off相同
 B 0 30  磁盘文件起始的30秒
 A -10 30  初动前10秒到初动后30秒
 B N 2048   磁盘文件最初的2048个点
 30.2 48  相对磁盘文件0点的30.2到48秒
\end{SACCode}
你也可以在CUTERR命令中使用FILLZ选项,在文件的开始或结尾处补0,这种做法适用于定义一个cut长度
大于文件长度的cut操作,然后使用READ命令读入内存。
\begin{SACCode}
SAC> r N11A.lhz
SAC> lh npts
FILE: N11A.lhz - 1
npts = 3101
SAC> cuterr fillz; cut b n 4096
SAC> r
SAC> lh npts
FILE: N11A.lhz - 1
npts = 4096
\end{SACCode}
 
\SACTitle{错误消息}
\begin{itemize}
\item[-]1322: 未定义文件剪裁的起始
	\begin{itemize}
  	\item[-]头段中的参考值未定义
	\item[-]这个错误可以用CUTERR命令控制
	\item[-]当这个错误被关闭时,使用磁盘文件起始值作为替代
	\end{itemize}
\item[-]1323: 未定义文件裁剪的结束值
\item[-]1324: 起始裁剪值小于文件开始值
	\begin{itemize}
	\item[-] 这个错误可以使用CUTERR来控制
	\item[-] 当这个错误被关闭时,使用文件的起始值代替错误值或者在文件起始补0
	\end{itemize}
\item[-]1325: 结束裁剪值大于文件结束值
\item[-]1326: 起始裁剪值大于文件结束值
	\begin{itemize}
	\item[-]错误的cut参数
	\item[-]这个错误不可关闭
	\end{itemize}
\end{itemize}

注意: 由于这是一个参数设定类型的命令,因而上面的错误在执行read命令之前是不会出现的。
当然上面的一些错误也可以通过cuterr命令转换为警告。

\SACTitle{限制}
目前不支持非等间隔文件或谱文件的截断。该命令对ASCII格式的SAC文件无效。

\SACTitle{相关命令}
READ, APK, PLOTPK, SYNCHRONIZE, CUTERR

\SACTitle{最近修订}
April 21, 2010 (Version 101.4)
