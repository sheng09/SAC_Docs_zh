\SACCMD{tsize}
\label{cmd:tsize}

\SACTitle{概要}
控制文本尺寸属性

\SACTitle{语法}
TSIZE [ Tiny | Small | Medium | Large v ] [ RATIO v ] [ OLD | NEW ]

\SACTitle{输入}
\begin{itemize}
\item size v : 改变文本尺寸之一的值为v 
\item RATIO v : 改变文本的宽高比为v 
\item OLD : 将所有文本尺寸值设置为其旧值,这些旧值用在SAC 9之前的版本中 
\item NEW : 改变所有文本尺寸值为SAC初始化时的缺省值。
\end{itemize}

\SACTitle{缺省值}
TSIZE RATIO 1.0 NEW

\SACTitle{说明}
大多数的文本注释命令(TITLE, XLABEL, FILEID等)允许你改变要显示的文本的尺寸。你可以从四个已命名的尺寸中选择(TINY, SMALL, MEDIUM, 和LARGE.)。每一个命名尺寸有一个初始值,如下表所示。其尺寸是一个字符相对整个视窗的高度。有些时候你想要使用一些不同与默认尺寸的注释。TSIZE允许你重新定义这四个已命令的尺寸。你也可以使用这个命令改变字符的宽-高比。
\begin{center}
\begin{tabular}{lccccc}
\toprule
NAME	&	A	&	B	&	C	&	D	&	E	\\
\midrule
TINY 	& 0.015 &   66 	&  50  	&	68  &	110	\\
SMALL	& 0.020 &	50  &  37  	&	66  &	82	\\
MEDIUM  & 0.030 &	33  &  25  	&	44  &	55	\\
LARGE	& 0.040 &	25  &  18  	&	33  &	41	\\
\bottomrule
\end{tabular}
\end{center}

上面做各列的定义如下:
\begin{itemize}
\item A 自如相对整个视窗的高度
\item B 全视窗下文本的行数
\item C 正常视窗下文本的行数。正常视窗是指x为0.到1.,y为0.到0.75
\item D 正常视窗中,每行的最小字符数
\item E 正常视窗中每行字符的平均数
\end{itemize}

\SACTitle{例子}
为了改变MEDIUM的定义,并使用它创建一个特别尺寸的标题:
\begin{SACCode}
SAC> TSIZE MEDIUM 0.35
SAC> TITLE 'Rayleigh Wave Spectra' SIZE MEDIUM
SAC> PLOT2
\end{SACCode}

为了重置尺寸定义到其缺省值:
\begin{SACCode}
SAC> TSIZE NEW
\end{SACCode}

\SACTitle{相关命令}
TITLE, XLABEL, FILEID, PLOTC

\SACTitle{最近修订}
July 22, 1991 (Version 9.1)

