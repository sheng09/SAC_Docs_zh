\SACCMD{fft}
\label{cmd:fft}

\SACTitle{概要}
进行离散Fourier变换

\SACTitle{语法}
\begin{SACSTX}
FFT [WO!MEAN!|W!MEAN!] [R!LIM!|A!MPH!]
\end{SACSTX}

\SACTitle{输入}
\begin{itemize}
\item WOMEAN :  变换之前去除均值 
\item WMEAN : 变换中保留均值 
\item RLIM : 输出为实部-虚部格式
\item AMPH : 输出为振幅-相位格式
\end{itemize}

\SACTitle{缺省值}
\begin{SACDFT}
fft wmean amph
\end{SACDFT}

\SACTitle{说明}
在变换进行之前,每个数据文件都要补零,使得数据点数为2的整数次幂。在磁盘和内存中的数据文件,包括时间序列数据和谱数据。谱数据可以是振幅-相位格式或实部-虚部格式。头段变量IFTYPE告诉你文件是什么格式的。多数命令只对一种文件类型起	作用。像FFT,IFFT,UNWRAP这样的命令将内存中的文件从一种文件格式转换为另一种格式。这个命令产生的谱文件可以用plotsp绘图、或者使用write或writesp命令保存在磁盘上。如果内存中有多于余个文件,PLOT2可以用于绘制谱文件振幅或者实部。

\SACTitle{头段变量改变}
在变换时B, E 和DELTA被改成起始频率、结束频率以及采样频率。B, E, NPTS和DELTA的初始值被保存在SB, SE, NSNPTS和SDELTA,如果做反Fourier变换这些值会被送回

\SACTitle{错误消息}
\begin{itemize}
\item[-]1301: 未读入文件
\item[-]1306: 对非等间隔数据的非法操作
\item[-]1307: 对谱文件非法操作
\item[-]1606: 超过DFT的最大允许数据点数
\end{itemize}

\SACTitle{限制}
变换最大数据点数为$2^{24}=16777216$

\SACTitle{相关命令}
\nameref{cmd:plotsp}、\nameref{cmd:ifft}、\nameref{cmd:writesp}
