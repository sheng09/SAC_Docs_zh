\section{arraymap}
\label{cmd:arraymap}

\SACTitle{概要}
利用SAC内存中的所有文件产生一个台阵或联合台阵的分布图

\SACTitle{语法}
ARRAYMAP [Array|Coarray]

\SACTitle{输入}
\begin{itemize}
\item ARRAY: 这个选项根据头段变量中的偏移X、Y值绘制台站分布
\item COARRAY: 这个选项根据各台站之间的相对坐标绘制台站分布图
\end{itemize}

\SACTitle{缺省值}
ARRAYMAP ARRAY

\SACTitle{头段数据}
下面的两个头段变量必须使用SAC宏文件wrxyz或者与之功能相似的其他函数提前设定,所有的偏移是相对于某个参考点的千米数。

USER7: 向东的偏移(x).

USER8: 向北的偏移(y).

这个命令中不使用向上的偏移

\SACTitle{说明}
不是很清楚这个命令的作用是什么,对于每个数据来说,需要用宏文件wrxyz定义头段
变量USER7和USER8,然后才能利用该命令绘制出arraymap,从命令的名字来理解,应该
是绘制某个台站的台站分布图,理论上只需要台站的真实位置即可。不知这个究竟在什
么场合要使用。

\SACTitle{限制}
在BBFK中允许的最多台站数

\SACTitle{相关命令}
WRXYZ是一个SAC宏文件,位于SACAUX/macros中

\SACTitle{最近修订}
July 22, 1991 (Version 10.5c)
