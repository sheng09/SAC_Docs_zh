\section{whpf}
\label{cmd:whpf}

\SACTitle{概要}
将辅助内容写入HYPO格式的震相拾取文件中

\SACTitle{语法}
WHPF IC n m

\SACTitle{输入}
\begin{itemize}
\item IC n m  在第18和19列插入带有两个整数的n和m的指令卡。n的允许值为0、1、5,m的允许值为0、1、9。
\end{itemize}

\SACTitle{说明}
``指令卡''用于分开在HYPO文件中的不同事件,参见HYPO71手册。关闭一个已经打开的HYPO震相拾取文件(CHPF)或者退出SAC时,将自动添加``10''指令卡到震相读取文件中。

\SACTitle{错误消息}
\begin{itemize}
\item[-]1908: HYPO震相拾取文件未打开
\end{itemize}

\SACTitle{相关命令}
CHPF, OHPF

\SACTitle{参考文献}
W.H.K. Lee and J.C. Lahr; HYPO71 (Revised): A Computer Program for Determining 	Hypocenter, Magnitude, and First Motion Pattern of Local Earthquakes; U. S. Geological 	Survey report 75-311.

\SACTitle{最近修订}
March 20,1992 (Version 10.6e)
