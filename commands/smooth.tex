\SACCMD{smooth}
\label{cmd:smooth}

\SACTitle{概要}
对数据应用算术平滑算法

\SACTitle{语法}
\begin{SACSTX}
SMOOTH [MEAN|MEDIAN] [H!ALFWIDTH! n]
\end{SACSTX}

\SACTitle{输入}
\begin{itemize}
\item MEAN : 应用均值平均算法 
\item MEDIAN : 应用中值平滑算法 
\item HALFWIDTH n : 设置移动窗的半宽为n。滑动窗将包含要平滑的数据点的前面和后面各n点 
\end{itemize}

\SACTitle{缺省值}
\begin{SACDFT}
smooth mean halfwidth 1
\end{SACDFT}

\SACTitle{说明}
这个命令对每个数据点应用算术平滑算法。算法类型和围绕每个数据点的滑动窗尺寸是可以变化的。每个窗的尺寸由指定其半宽度来定义,使其滑动窗以每个数据点为中心,并使窗的长度为奇数个点,这样使算法更加简单易行,且不会引起歧义。

\SACTitle{头段变量改变}
depmin, depmax,depmen
