\SACCMD{whiten}
\label{cmd:whiten}

\SACTitle{概要}
平滑输入的时间序列的频谱。

\SACTitle{语法}
\begin{SACSTX}
W!H!IT!EN! n [F!ILTER!D!ESIGN!]
\end{SACSTX}

\SACTitle{输入}
\begin{itemize}
\item n :  阶数(极数)。这个数越大,结果数据就越平滑。高阶可以更好的清除一些数据,但是也可能会导致对数据处理过多而丢掉一些重要的数据。默认值为6。 
\item FD : 进行一些类似于filterdesign的命令,使用白化系数,设计一个白化滤波器。详情可以参考filterdesign命令。
\end{itemize}

\SACTitle{缺省值}
\begin{SACDFT}
whiten 6
\end{SACDFT}


\SACTitle{说明}
对数据中加入白噪声。平滑输入时间序列的频谱。当这个命令在谱分析命令(比如子程序SPE中的命令、transfer或spectrogram)之前执行,其减少了频谱值的动态范围,提供了对地震数据高频操作的精度。

WHITEN可以在SPE子程序内部调用,或者从SAC的主shell中调用。SPE中的WHITEN和主shell中的WWHITEN分别有不同的阶数。在主shell中,你可以调用WHITEN 4,下一次在主shell中调用WHITEN时阶数为4,但是在SPE中调用WHITEN时依然是缺省的6阶,除非你在SPE的命令行中进行了修改。进一步,SPE中的阶数与SPE COR命令的PREWHITEN选项是一样的(设置了一个其他的也就设置了)。当然主shell中WHITEN命令与TRANSFER命令中的PREWHITEN选项也是一样的。

\SACTitle{相关命令}
\nameref{cmd:transfer}
