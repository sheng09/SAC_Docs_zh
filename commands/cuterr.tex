\SACCMD{cuterr}
\label{cmd:cuterr}

\SACTitle{概要}
控制由于坏的截窗参数引起的错误

\SACTitle{语法}
\begin{SACSTX}
CUTERR FA!TAL!|U!SEBE!|FI!LLZ!
\end{SACSTX}

\SACTitle{输入}
\begin{itemize}
\item FATAL : 将裁剪错误设置为致命
\item USEBE : 将坏的起始和结束裁剪参数设置为文件开始和文件结束 
\item FILLZ : 在文件开始时间之前或文件结束时间之后填补适当数目的0以弥补坏的裁剪参数.
\end{itemize}

\SACTitle{缺省值}
对于信号迭加子程序默认值为FILLZ,其他的默认值为USEBE

\SACTitle{说明}
这个命令控制由于坏的裁剪参数引起的错误条件。其可以定义为致命错误(FATAL)。如果裁剪参数的
起始值或结束值在文件头段中未定义,则可以选择为USEBE。如果裁剪参数有定义但是其裁剪起始值
小于文件起始值或者裁剪参数结束值大于文件结束值,则可以分别用文件起始结束值代替裁剪参数,
或者也可以使用FILLZ在文件前后补适当的0

\SACTitle{例子}
假设文件FILE1起始时间为B=25s,初动到时A=40s,采样率为0.01s。
\begin{SACCode}
SAC> cut a -20 e
SAC> read file1
\end{SACCode}
裁剪起始值为20s,产生了一个错误条件。在USEBE模式下,裁剪起始值将替换为25s(即B)。在FILLZ模式下,
在数据之前将将插入500个0(5秒钟,每秒100个点),裁剪起始值保持为20s。

\SACTitle{相关命令}
\nameref{cmd:cut}、\nameref{cmd:read}
