\section{trace}
\label{cmd:trace}

\SACTitle{概要}
跟踪暂存块变量和头段变量

\SACTitle{语法}
TRACE [ ON | OFF ] name [ name ... ]

\SACTitle{输入}
\begin{itemize}
\item ON : 打开后随变量的跟踪 
\item OFF : 关闭后随变量的跟踪 
\item name : 要跟踪的暂存块变量或头段变量名。如果是一个头段,则其形式为filename,hdrname。这里filename是SAC数据文件名或文件号,hdrname是SAC头段变量名 
\end{itemize}

\SACTitle{缺省值}
TRACE ON

\SACTitle{说明}
这个命令用于在SAC执行过程中跟踪SAC暂存块变量或头段变量的值。其对于调试长的或复杂的宏文件很有用。对一个变量进行跟踪时,将显示其当前值。跟踪选项打开时,个命令执行后都对变量值进行检查。每次暂存块变量或头段变量改变时都将显示一行的信息。跟踪选项关闭时,也最后一次显示它的当前值。

\SACTitle{例子}
跟踪暂存块变量TEMP1和属于文件MYFILE的头段变量T0:
\begin{SACCode}
SAC> TRACE ON TEMP1 MYFILE,T0
  TRACE  (on) TEMP1 = 1.45623
  TRACE  (on) MYFILE,T0 = UNDEFINED
\end{SACCode}

执行命令时,不论是在终端键入命令还是执行一个宏文件,SAC都将检查新的变量值和内存中的变量值,并且不论哪个值改变,都将有关的信息显示出来。假设在完成一些计算之后改变了TEMP1,并定义了T0的值,则SAC将显示信息:
\begin{SACCode}
  TRACE (mod) TEMP1 = 2.34293
  TRACE (mod) MYFILE,T0 = 10.3451
\end{SACCode}

稍后的处理中TEMP1可能再次改变:
\begin{SACCode}
  TRACE (mod) TEMP1 = 1.93242
\end{SACCode}

当跟踪选项关闭时,SAC最后一次显示当前值:
\begin{SACCode}
SAC> TRACE OFF TEMP1 MYFILE,T0
  TRACE (off) TEMP1 = 1.93242
  TRACE (off) MYFILE,T0 = 10.3451
\end{SACCode}

\SACTitle{最近修订}
January 27, 1989 (Version 10.4B)
