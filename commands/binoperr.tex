\section{binoperr}
\label{cmd:binoperr}

\SACTitle{概要}
控制发生在二进制文件操作中的错误

\SACTitle{语法}
BINOPERR [Npts Fatal|Warning|Ignore] [Delta Fatal|Warning|Ignore]

\SACTitle{输入}
\begin{itemize}
\item Npts: 改变由于数据点不相等的错误条件 
\item Delta: 改变由于采样间隔不相等的错误条件 
\item Fatal: 是错误条件为致命性条件。一旦发生错误,控制立即返回终端,忽略剩余的全部命令。
\item Warning: 给用户发送警告消息,校正错误条件并继续执行
\item Ignore: 纠正错误条件继续执行
\end{itemize}

\SACTitle{缺省值}
BINOPERR NPTS FATAL DELTA FATAL

\SACTitle{说明}
无论你执行什么二元运算模块中的命令(ADDF, DIVF等),SAC都会检测一些共同的错误。使用这个命令,你可以控制SAC在遇到这些错误时如何处理。如果你设置错误条件为致命,那么SAC在遇到这种错误时将停止执行当前命令,忽略接下来的所有命令,打印错误信息到终端并将控制权交还。如果设置错误条件为警告,则发送一个警告消息,尽可能纠正错误并继续执行。如果设置错误条件为忽略,则SAC会纠正错误并继续而不告诉你发生了什么。

对于要操作的两个数据文件数据点数不等的错误,SAC会使用数据点最少的那个文件	的数据点数作为参考,以保证正常操作。

另一种错误出现在两个要操作的文件采样间隔不等,则SAC会使用第一个数据文件的	采样间隔作为结果文件的采样间隔。

\SACTitle{例子}
假定FILE1有1000个数据点,FILE2有950个数据点::
\begin{SACCode}
SAC> BINOPERR NPTS FATAL
SAC> READ FILE1
SAC> ADDF FILE2
ERROR:  Header field mismatch: NPTS FILE1 FILE2
\end{SACCode}
上例文件加法未执行,假设你输入:
\begin{SACCode}
SAC> BINOPERR NPTS WARNING
SAC> ADDF FILE2
WARNING:  Header field mismatch: NPTS FILE1 FILE2
\end{SACCode}
文件加法对每个文件的前50个数据执行了加法操作

\SACTitle{相关命令}
ADDF, SUBF, MULF, DIVF

\SACTitle{最近修订}
January 8, 1983 (Version 8.0)
