\section{plabel}
\label{cmd:plabel}

\SACTitle{概要}
定义一般绘图的标签以及它们的属性

\SACTitle{语法}
PLABEL [ n ] [ ON | OFF | text ] [ SIZE Tiny | Small | Medium | Large ] [ BELOW | POSITION x y [ a ] ]

\SACTitle{输入}
\begin{itemize}
\item n : 绘图标签号目前有5个,如果省略,则为上一个标签号加1
\item ON : 打开绘图标签选项 
\item OFF : 关闭绘图标签选项 
\item text : 改变绘图标签的文本内容,同时打开了绘图标签选项 
\item SIZE size :  改变绘图标签的尺寸 
\item TINY : 微小尺寸,每行132个字符 
\item SMALL :  小尺寸,每行100个字符 
\item MEDIUM : 中等尺寸,每行80字符 
\item LARGE : 大尺寸,每行50字符 
\item BELOW : 将这个标签放在以前的标签的下面 
\item POSITION x y a : 定义该标签的位置,其中x的取值为0~1,y的取值为0到最大视口(一般为0.75),a是标签相对于水平线的顺时针旋转的角度
\end{itemize}

\SACTitle{缺省值}
默认字体大小为small,标签1的位置为0.15 0.2 0. 
 
默认其他标签的位置为上一个标签之下

\SACTitle{说明}
这个命令允许你定义一般用途的绘图标签用于后面的绘图命令,你可以定义每个标签的位置。文本质量以及字体可以用GTEXT命令设定,也可以使用TITLE、XLABEL、YLABEL生成图形的标题以及轴标签。

\SACTitle{例子}
下面的命令将在接下来的绘图中在左上角产生一个四行的标签:
\begin{SACCode}
SAC> PLABEL 'Sample seismogram' POSITION .12 .5
SAC> PLABEL 'from earthquake'
SAC> PLABEL 'on January 24, 1980'
SAC> PLABEL 'in Livermore Valley, CA'
\end{SACCode}

一个额外的小标签可以放在左下角:
\begin{SACCode}
SAC> PLABEL 5 'LLNL station: CDV' S T P .12 .12
\end{SACCode}

\SACTitle{相关命令}
GTEXT , TITLE , XLABEL , YLABEL

\SACTitle{最近修订}
July 22, 1991 (Version 9.1)
