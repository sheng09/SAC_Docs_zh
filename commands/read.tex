\SACCMD{read}
\label{cmd:read}

\SACTitle{概要}
从磁盘读取SAC文件到内存

\SACTitle{语法}
\begin{SACSTX}
R!EAD! [MORE] [TRUST ON|OFF] [DIR CURRENT|name] [XDR|ALPHA|SEGY] 
    [SCALE ON|OFF] [filelist]
\end{SACSTX}
所有的选项必须位于filelist之前。

\SACTitle{输入}
\begin{itemize}
\item MORE : 将新文件添加到内存中老文件之后,如果这个选项忽略,则新数据将替代老数据。
\item TRUST ON|OFF : 这个选项用于解决将文件从SAC转换到CSS格式过程中出现的问题。当转换数据时,匹配的事件id意味着文件含有相同的事件信息,或者可能是两个不同格式合并之后出现的干扰。当TRUST打开时,SAC将更有可能接受匹配事件id作为相同的事件信息,这依赖于READ命令与当前内存中数据的历史。
\item DIR CURRENT : 从当前目录读取所有简单文件名,这个目录是你启动SAC的目录 
\item DIR name : 从目录name中读取全部简单文件,其可以为绝对/相对路径 
\item XDR : 输入的文件是XDR格式,这个格式用于实现不同构架的二进制数据的转换 
\item ALPHA : 输入文件是SAC格式的字符数字型文件,ALPHA与XDR选项不兼容 
\item SEGY : 读取IRIS/PASSCAL定义的SEGY格式文件,这种格式允许一个文件包含一个波形 
\item SCALE : 只和SEGY选项搭配使用,SCALE选项默认是关的。当SCALE选项为OFF时,SAC从SEGY中读取counts,当SCALE为ON时,SAC将counts乘以文件给出的一个SCALE因子。如果SCALE为OFF,则这个文件中的SCALE值将储存在SAC头段SCALE中,如果SCALE为ON,SAC的SCALE字段将被设置为1.0,SCALE由于仪器响应的原因是一个粗糙的方法。更好的方法是使用TRANSFER命令。最好在READ中不要使用SCALE选项,而是直接使用TRANSFER命令。SCALE只应在TRANSFER命令需要的响应信息无法获得时才使用 
\item filelist :  file | wild 
\item file : 一个文件名。其可以是简单文件名或路径名。路径名可以是相对/绝对路径。
\item wild : 通配符以扩展文件名列表 
\end{itemize}

\SACTitle{缺省值}
\begin{SACDFT}
read commit dir current
\end{SACDFT}

\SACTitle{说明}
SAC几乎所有命令只对当前内存中的文件操作。内存中的数据类似于文本编辑器使用的临时工作文件。READ命令从一个或多个磁盘文件传递数据到内存。默认是读取文件中全部数据到内存。

CUT命令可以用于指定只有磁盘文件的一部分要被读入。SAC文件在2000年之后产生的文件被假定年的值为四位数字。年份为两个数字的文件被假定为20世纪,会被加上1900。正常情况下内存中原先的数据会由于另一个READ命令的执行而丢失。新数据将取代老数据。如果关键字MORE是命令的第二个符号,则新数据将放在内存中老数据的后面。数据文件列表称为老文件列表和新文件列表的连接。在三种情况下MORE项可能有用:
\begin{itemize}
\item 文件列表太长在一行下无法打印出
\item 在长文件列表中一个文件名拼错
\item 一个文件被读入,作了些处理,然后与原始数据比较
\end{itemize}
文件名可以包含通配符,你可以使用他们去匹配单个字符、多个字符等等。

\SACTitle{例子}
read命令的简单示例位于~``\nameref{sec:read-and-write}''~一节。

现在假设你启动SAC的位于目录``/me/data''下,你想要处理子目录``event1''和``event2''下的文件:
\begin{SACCode}
SAC> read dir event1 f01 f02
\end{SACCode}
读取了目录/me/data/event1下的文件:
\begin{SACCode}
SAC> read f03 g03
\end{SACCode}
相同目录下的文件被读入:
\begin{SACCode}
SAC> read dir event2 *
\end{SACCode}
/me/data/event2下的全部文件被读入:
\begin{SACCode}
SAC> read dir current f03 g03
\end{SACCode}
目录/me/data下的文件被读入。

\SACTitle{错误消息}
\begin{itemize}
\item[-]1301: 未读入数据文件(未给出要读取的文件列表或文件列表中文件均不可读)
\item[-]1320: 可用内存不足以读取文件
\item[-]1314: 数据文件列表不得以数字开头
\item[-]1315: 文件列表的最大数目为1000
\item[-]6002: 没有可用数据集
\end{itemize}

\SACTitle{警告消息}
\begin{itemize}
\item[-]0101: 打开文件
\item[-]0108: 文件不存在
\item[-]0114: 读取文件
\end{itemize}

\SACTitle{头段变量改变}
e, depmin, depmax, depmen, b 

\SACTitle{相关命令}
\nameref{cmd:cut}、\nameref{cmd:readerr}、\nameref{cmd:wild}
