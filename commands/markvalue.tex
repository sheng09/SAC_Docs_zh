\SACCMD{markvalue}
\label{cmd:markvalue}

\SACTitle{概要}
在数据文件中搜索并标记某个值

\SACTitle{语法}
MARKVALUE [ GE | LE v ] [ TO marker ]

\SACTitle{输入}
\begin{itemize}
\item GE v : 搜索并标记第一个大于或等于v的数据点 
\item LE v : 搜索并标记第一个小于或等于v的数据点 
\item TO marker : 定义头段中储存结果的时间标记 
\item marker : T0|T1|T2|T3|T4|T5|T6|T7|T8|T9 
\end{itemize}

\SACTitle{缺省值}
MARKVALUE GE 1 TO T0

\SACTitle{说明}
这个命令在每一个数据文件中搜索,满足要求的值并将第一次出现该值的时间标记下来,如果测量时间窗已经被定义(参见MTW),则只有那部分的文件才会被搜索。否则将搜索整个文件,结构写入头段中的时间标记中

\SACTitle{例子}
搜索文件中第一个值大于3.4的点并将结果保存在头段T7中:
\begin{SACCode}
SAC> MARKVALUE GE 3.4 TO T7
\end{SACCode}

稍后在以T4起始的10s测量时间窗中使用相同的搜索:
\begin{SACCode}
SAC> MTW T4 0 10
SAC> MARKVALUE
\end{SACCode}

\SACTitle{头段变量改变}
Tn, KTn

\SACTitle{相关命令}
MTW

\SACTitle{最近修订}
May 15, 1987 (Version 10.2)
