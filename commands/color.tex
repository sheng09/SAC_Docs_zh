\section{color}
\label{cmd:color}

\SACTitle{概要}
控制彩色图形设备的颜色选项

\SACTitle{语法}
COLor [ON|OFF|color] [Increment [ON|OFF]] [Skeleton color] [Background color] [List Standard|colorlist]

color是下面中的一个:

White|Red|Green|Yellow|BLUe|Magenta|Cyan|BLAck
		
*这里有些参数在缩写的情况下可能会有歧义,请谨慎使用,而且LIST选项必须放在命令的最后

\SACTitle{输入}
\begin{itemize}
\item color: 颜色表中标准颜色名或颜色的数字代码
\item COLOR ON : 打开颜色选项单数不改变其他选项
\item COLOR OFF: 关闭颜色选项
\item COLOR color : 打开颜色选项并将数据设置为颜色color
\item Increment ON: 每个数据文件绘出后,根据colorlist的顺序改变颜色
\item Increment OFF: 不改变数据颜色 
\item Skeleton color: 按照标准颜色名或颜色号修改边框颜色
\item Background color: 修改背景色为color
\item LIST colorlist: 改变颜色列表,将数据颜色设置为列表中第一个颜色,并打开颜色开关
\item LIST Standard: 将颜色列表设为标准列表,将数据颜色设置为列表中第一个颜色,并打开颜色开关
\end{itemize}

\SACTitle{缺省值}
COLOR BLACK INCREMENT OFF SKELETON BLACK BACKGROUND WHITE LIST STANDARD

\SACTitle{说明}
这个命令控制设备的颜色属性,数据颜色是用于绘制这个数据文件的颜色。当一个数据文件绘制
完毕后,数据颜色可以根据颜色列表自动改变。skeleton颜色是用于绘制注释轴、标题、网格 、
框架的颜色。背景色是空框架在未绘制任何图形之前的颜色。

大多数时间你会选择标准颜色名,比如red,这是与图形设备无关的。然而有时候你可能想选择一个
非标准颜色,比如aquamarine,这个可以将颜色表装入图形设备来实现。

这个表将特定的颜色、亮度、对比度等与一个数字联系起来,然后你就可以通过设定对应的整数值
选择aquamarine作为你的绘图的一个部分的颜色,这个需要点工作量,可是如果你喜欢,这就值得。

如果你正在同一张图上绘制多个数据文件,通过INCRMENT选项可以使得不同数据有不同的颜色。
标准颜色表顺序如下:

RED, GREEN, BLUE, YELLOW, CYAN, MAGENTA, BLACK

\SACTitle{例子}
为了使数据颜色从红色开始不断变换:
\begin{SACCode}
SAC> color red increment
\end{SACCode}
为了设置数据颜色为红色,背景白色,蓝色边框:
\begin{SACCode}
SAC> color red background white skeleton blue
\end{SACCode}
为了设置一个数据颜色不断变换,颜色列表为red,white,blue,背景色为aquamarine(!!!):
\begin{SACCode}
SAC> color red increment backgroud 47 list red white blue
\end{SACCode}
上面的例子假设aquamarine是颜色表的47号。

*目前背景色不可用*

\SACTitle{最近修订}
April 13, 1987 (Version 10.1)
