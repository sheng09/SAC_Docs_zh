\SACCMD{saveimg}
\label{cmd:saveimg}

\SACTitle{概要}
将图像显示窗口的中图像保存到多种格式的图形文件中

\SACTitle{语法}
\begin{SACSTX}
SAVEIMG filename.format
\end{SACSTX}

\SACTitle{输入}
\begin{itemize}
\item filename : 要保存的图像文件名。 
\item format : 图像文件格式,支持PS、PDF、PNG和XPM。
\end{itemize}

\SACTitle{说明}
这个命令将当前绘图保存到图像文件中,可用的格式包Postscript(ps),Portable Document Format(pdf),Image file(png)以及Pixmap file(xpm)。

SAVEIMG相对于SGF文件的好处在于,SGF文件中的字母和数字是由线段组成的,而SAVEIMG产生的ps或pdf图像采用Postscript 特性直接产生字体。对大多数情况,低精度的.png或.xpn文件也能满足要求。
.png和.xpm将拥有当前窗口的横纵比(参见WINDOW命令),.pdf或.ps文件拥有固定的横纵比X/Y=11/8.5=1.2941,对这些绘图,如果显示窗口设置为1.2941会看起来比较好。

正如.sgf文件,绘图没有边框。对于.sgf文件,脚本sac/bin/sgftoeps.csh可以产生一个有边框的.eps文件(要求程序Ghostscript在路径中)。相似的脚本未来会写出来应用于SAVEIMG。

为了使用SAVEIMG保存一个绘图,图像必须是可见的,即通过PLOT、PLOT1等命令绘制出来。SAVEIMG在子程序SSS中无法工作,但如果输入qs退出子程序,此时图像窗口未关闭,SAVEIMG此时可用于保存该图像。

\SACTitle{例子}
将图像保存为PDF文件:
\begin{SACCode}
SAC> read PAS.CI.BHZ.sac
SAC> p1
SAC> saveimg pas.ci.pdf
\end{SACCode}

将谱图用多种格式保存:
\begin{SACCode}
SAC> fg seismo
SAC> spectrogram
SAC> save spectrogram.ps
SAC> save spectrogram.png
SAC> save spectrogram.pdf
\end{SACCode}
