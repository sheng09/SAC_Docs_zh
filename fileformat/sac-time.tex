\section{SAC中的时间概念}
SAC头段中与时间有关的变量很多,大致分为两类。一类为绝对时间,变量包括NZYEAR,NZJDAY,
NZHOUR,NZMIN,NZSEC,NZMSEC,这六个变量决定了某年某日的某个时刻,一般用与之等效的
KZDATE和KZTIME\footnote{这两个变量虽然说是头段变量,其实在文件中并不占据存储空间,
其由六个时间变量衍生得到}来表示,其构成了SAC中唯一一个绝对时间,在SAC中称为参考时间或
零时间。你可以任意设置,通常其为第一个数据点所对应的时间,但是也可以为事件的发
生时间\footnote{选择发震时刻作为参考时间的一个好处在于,可以很方便的用头段变量中相对
时间来表示走时}、某个午夜或你的生日。另一类为相对时间,变量包括B、E、O、A、F、Tn(n=0-9),
这几个相对时间变量的值都是相对于参考时间的秒数,根据参考时间以及各个时间变量的相对值即
可以确定该数据中任意点的绝对时间。在我的一篇\href{http://seisman.info/timing-of-sac.html}{博文}
\footnote{\url{http://seisman.info/timing-of-sac.html}}中有关于SAC时间
的更细致的测试和讨论,那里将时间变量分为三类,在你理解了现在的说法后可以看看那篇博文。
