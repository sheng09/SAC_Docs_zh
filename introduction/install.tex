\section{安装SAC}
\label{sec:sac-install}
本节介绍如何在Linux下安装SAC,要求读者了解Linux的一些基本概念和操作。

\subsection*{申请SAC}
在``\nameref{sec:history}''中已经说到,SAC协议仅允许在IRIS社区内部分享SAC的源码,
所以SAC不像很多软件一样可以很方便地通过软件包管理器安装或者直接从网络下载源码。

软件包申请地址~:~\url{http://www.iris.edu/ds/nodes/dmc/forms/sac/}

认真填写个人信息,尤其注意Email那一栏,需要填写单位邮箱,如果使用
QQ、163这样的邮箱很容易直接被拒绝,如果没有单位邮箱,需要提供其它
信息以验证你的身份。

IRIS提供了SAC源码包、Linux 64位二进制包和Mac 64位二进制包。对于Linux
用户,可以通过``\verb+uname -a+''命令查看当前系统是32位还是64位。对于64位系统,
可以直接使用Linux 64位二进制包;对于Linux 32位系统,则必须手动编译SAC源码。

考虑到在申请提交之后,需要人工审核,两三个工作日后才会通过邮箱获取SAC软件包,
建议还是同时申请Linux 64位二进制包和SAC源码包。

\subsection*{安装依赖包}
Linux下安装软件最麻烦的一个问题就是软件之间的依赖关系。

对于Ubuntu/Debian系\footnote{很久不用Ubuntu,无法保证完全正确。}:
\begin{minted}{console}
$ sudo apt-get install build-essential
$ sudo apt-get install libncurses5-dev libsm-dev libice-dev
$ sudo apt-get install libxpm-dev libx11-dev zlib1g-dev
\end{minted}

对于CentOS/Fedora/RHEL系:
\begin{minted}{console}
$ sudo yum groupinstall 'Development Tools'
$ sudo yum install glibc ncurses-devel libSM-devel libICE-devel
$ sudo yum install libXpm-devel libX11-devel zlib-devel
\end{minted}

\subsection*{安装}
安装的方法在这一步分成两个部分,分别是二进制包安装和源代码编译,根据实际情况
二者选一。
\subsubsection*{二进制包安装}
解压sac二进制包:
\begin{minted}{console}
$ tar -zxvf sac-101.6a-linux_x86_64.tar.gz
\end{minted}

复制sac文件夹到安装目录(推荐安装目录为\verb+/usr/local+):
%\begin{minted}{console}
%$ sudo cp -r sac /usr/local
%\end{minted}

\subsubsection*{源代码编译}
\begin{minted}{console}
$ tar -zxvf sac-101.6a_source.tar.gz
$ cd sac-101.6a
$ ./configure --prefix=/usr/local/sac
$ make
$ sudo make install
\end{minted}

\subsection*{配置变量}
向~\verb+~/.bashrc+~中加入如下语句以配置环境变量和SAC全局变量:
\begin{minted}{bash}
export SACHOME=/usr/local/sac
export SACAUX=$SACHOME/aux
export PATH=$SACHOME/bin:$PATH

export SAC_DISPLAY_COPYRIGHT=1
export SAC_PPK_LARGE_CROSSHAIRS=1
export SAC_USE_DATABASE=0
\end{minted}

其中,
\begin{itemize}
\item \verb+SACHOME+~为SAC的安装目录;
\item \verb+SACAUX+~中包含了SAC运行所需的辅助文件;
\item \verb+PATH+~为Linux系统环境变量;
\item \verb+SAC_DISPLAY_COPYRIGHT+~用于控制是否在启动SAC时显示版本和版权信息,
    一般设置为1。在脚本中多次调用SAC时会重复显示版本和版权信息,干扰脚本的正常输出,因而
    在脚本中一般将其值设置为0,设置方法可以参考``\nameref{sec:sac-bash}''、``\nameref{sec:sac-perl}''和``\nameref{sec:sac-python}''
    中的相关内容;
\item \verb+SAC_PPK_LARGE_CROSSHAIRS+~用于控制震相拾取过程中光标的大小,在
    ``\nameref{sec:phase-picking}''一节会具体说明;
\item \verb+SAC_USE_DATABASE+~用于控制是否允许将SAC格式转换为GSE 2.0格式;
    一般用不到该特性,故而设置其值为0;
\end{itemize}

修改完~\verb+~/.bashrc+~后,执行以下命令使配置的环境变量生效:
\begin{minted}{console}
$ source ~/.bashrc
\end{minted}

\subsection*{启动SAC}
终端键入小写的sac\footnote{Ubuntu的源里有一个名叫sac的软件,是用来显示登录账户
的一些信息;CentOS的源里也有一个名叫sac的软件,是CSS语法分析器的Java接口。所以
一定不要试图用发行版自带的软件包管理器安装sac!!!},显示如下则表示SAC安装成功:
\begin{minted}{console}
$ sac
 SEISMIC ANALYSIS CODE [11/11/2013 (Version 101.6a)]
 Copyright 1995 Regents of the University of California

SAC>
\end{minted}
