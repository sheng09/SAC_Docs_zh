\section*{\centering 写在第三版前的一些废话}

\begin{quote}
工欲善其事,必先利其器。
\begin{flushright}
--- 《论语 $\cdot$ 卫灵公》
\end{flushright}
\end{quote}

2010年10月,刚刚大三没多久,开始接触并学习SAC;2011年的暑假
开始着手SAC文档的翻译工作;2012年01月,文档的v1.0版本发布;2013年03月
,文档的v2.0版发布。3年多的时间过去了,文档更新到了v3.0版。

整理完v2.0之后,我以为接下来的工作就只是修订一些拼写错误、增加一些对命令
细节的描述,文档的整体不会有太大的调整。目前看来,事情的发展已经超出了我
的预期之外。

v2.0几乎完全遵循了SAC官方文档的结构,这样的结构使得想要对文档进行补充和
完善变得非常困难。

2013年11月,George Helffrich著的
``The Seismic Analysic Code : A Primer and User's Guide''一书出版了。v3.0
借鉴了该书的结构和布局,并加入了更多的内容。

整个文档分为如下几个部分:
\begin{description}
\item[SAC简介] 简单介绍SAC软件的相关信息;
\item[SAC基础] 不涉及具体命令,介绍在正式开始学习之前需要知道的基础知识;
\item[SAC文件格式] 详细介绍SAC文件格式以及头段变量;
\item[SAC数据处理] 介绍如何利用SAC命令进行数据处理和分析;
\item[SAC绘图] 介绍如何在SAC中绘图以及修改图件的细节;
\item[SAC宏] 介绍SAC编程语言;
\item[SAC与Bash] 用Bash替代SAC宏的功能;
\item[SAC与Perl] 用Perl脚本语言替代SAC宏的功能;
\item[SAC函数库] 如何在自己的程序中使用SAC提供的函数;
\item[SAC I/O] 如何自己写程序读写SAC文件;
\item[SAC相关工具] 介绍与SAC相关的一些工具;
\item[SAC命令] 详细介绍SAC的每个命令;
\end{description}

对本文档的内容有疑问,或发现任何错误、笔误,欢迎发邮件给我或者在GitHub
项目主页上提交Issue,也欢迎感兴趣的读者fork该项目,共同完善文档。

此文档仅供个人学习使用,希望不涉及版权问题。

\begin{flushleft}
博客: \url{http://seisman.info}                                         \\
文档发布页: \url{http://seisman.info/sac-manual.html}                   \\
项目主页: \url{https://github.com/seisman/SAC_Docs_zh}                  \\
联系方式: \url{seisman.info@gmail.com}  
\end{flushleft}

\begin{flushright}
作者~:~SeisMan \\
2014年03月25日
\end{flushright}
