\section{SAC绘图功能}
\SACTitle{图像输出设备:}
SAC目前支持三种图形输出设备。
\begin{itemize}
\item Xwindows,是多数系统支持的具有高分辨率的位图图形窗口系统,简称为X,
大概就是当你输入PLOT之后出来的那个小窗口。
\item SGF,即SAC图形文件,一个SGF文件包含了在任何图形设备上绘制单个图形
所必须的全部信息。每个绘图储存在单独的文件中,文件名为``Fnnn.SGF''格式,
其中``nnn''为绘图编号,起始为001。可以使用SGF命令控制文件的一些特性,
另外有程序sgftops可以将SGF文件转换为postscript,也有脚本可以将其转换为pdf格式,这个以后
再说。
\item PDF, PS, PNG, XPM\footnote{PNG和XPM是位图图像格式,其精度不够,
且依赖于其他函数库,不推荐使用,101.6的二进制版默认不支持PNG和XPM格式}:SAC自101.5版本开始
加入了新的命令SAVEIMG,可以直接将图形保存为PS和PDF格式。
\end{itemize}

\SACTitle{图形控制模块:}
图形控制模块控制整个绘图设备和窗口:
\begin{itemize}
\renewcommand\labelitemi{\dag}	% \usepackage{pifont}
\item BEGINDEVICES、ENDDEVICES 选择或不选择一个设备用于绘图
\item ERASE 擦除图形区域
\item VSPACE 控制图像的最大尺寸和比例
\item SGF 控制SGF设备的属性
\end{itemize}

\SACTitle{图形操作模块}
图形操作模块用于实际绘制图形:
\begin{itemize}
\renewcommand\labelitemi{\dag}	% \usepackage{pifont}
\item PLOT 将内存中的数据绘制出来,一张图上绘制一个信号
\item PLOT1 一张图上绘制多个信号,共用X轴
\item PLOT2 一张图上绘制多个信号,共用X和Y轴
\item PLOTPK 用于拾取震相、走时等等
\item PLOTPM 在一对信号上绘制质点运动图
\item FILEID、FILENUMBER、PICKS 控制文件ID、文件号、时间标记是否显示
\item SETDEVICVE 设置绘图默认的图形设备
\item PLOTC 在SAC图上做各种类型的标记以达到注释的目的
\item PLOTALPHA 从磁盘读入ASCII文件并绘图
\item PLOTDY 创建一个带有误差棒的图
\item PLOTXY 以一个文件为自变量,其他文件为因变量绘图
\item PRINT 输出内存中最近的sgf文件
\end{itemize}

\SACTitle{图形环境模块}
图形环境模块控制图形的具体细节:
\begin{itemize}
\renewcommand\labelitemi{\dag}	% \usepackage{pifont}
\item XLIM、YLIM 控制X、Y轴的范围
\item XVPORT、YPORT 控制图形位于图形窗口的位置
\item TITLE 指定标题
\item XLABLE、YLABLE X、Y轴标签
\item PLABLE 通用轴标签设定
\item LINE、SYMBOL、COLOR、TSIZE 控制线型、符号、颜色、文本尺寸
\item GTEXT 控制绘图中文本的质量以及字体
\item BEGINWINDOW 选择一个特定的窗口绘图
\item BEGINFRAME、ENDFRAME 关闭(启动)不同绘图之间的自动更新框架的动作
\item XLIN、XLOG、YLIN、YLOG 分别设置X、Y轴的为线性或者对数坐标
\item LINLIN、LINLOG、LOGLIN、LOGLOG 同时设置两个坐标的属性
\item XDIV、YDIV 控制X、Y轴的刻度间隔
\item XFUDGE、YFUDGE 控制X、Y轴的插入因子
\item AXES、TICKS 控制坐标轴、刻度的位置
\item GRID、XGRID、YGRID、BORDER 控制网格和边界的绘制
\end{itemize}
对于对数坐标还有很多命令:XFULL、YFULL、LOGLAB、FLOOR、LOADCRTABLE、WAIT、WIDTH、NULL。
另外还有一个命令QDP可以用于快速绘图。
