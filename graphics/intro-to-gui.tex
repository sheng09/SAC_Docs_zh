\section{图形设备}
SAC支持两种图形设备,分别是xwindows和sgf,默认的图形设备是xwindows。可以使用
\nameref{cmd:begindevices}~和~\nameref{cmd:enddevices}~命令开启/关闭指定的图形设备;
同时也可以使用~\nameref{cmd:setdevice}~命令设定默认的图形设备。

\subsection{xwindows}
xwindows即X Window System,也称为X11或X,是一种以位图方式显示的软件窗口系统。
几乎所有的现代操作系统都能支持与使用X,Linux下知名的桌面环境GNOME和KDE也都是以
X窗口系统为基础建构成的。

\begin{figure}[H]
\centering
\includegraphics[width=0.9\textwidth]{plot}
\caption{SAC绘图窗口}
\label{fig:plot}
\end{figure}

图~\ref{fig:plot}~展示了SAC中的xwindows图形设备的外观,它是SAC默认的图形设备。
同很多其它软件界面类似,xwindows窗口在左上角显示图标,右上角显示``最小化''、
``最大化''、``还原''和``关闭''按钮。窗口的中间部分为真正的绘图区,
本文档的其余插图将只给出绘图区的图像而不再包含窗口部分。

左上角的``Graphics Window: 1''指明了当前绘图窗口的编号为``1'',
SAC最多支持同时打开10个X窗口,编号为1-10。默认情况下只启动并使用1号X窗口。
~\nameref{cmd:beginwindow}~命令用于启动指定编号的X窗口;
~\nameref{cmd:window}~命令还可以设置每个X窗口的长宽比以及X窗口相对于屏幕的位置。

\subsection{sgf}
SGF,全称SAC Graphic File,即SAC图形文件,是SAC自定义的一种文件格式,其包含了
绘制一个图件所需要的全部信息,可以通过~\nameref{sec:sgftops}~等工具转换到
其它图形设备或图形文件格式。

若启用了SGF图形设备,每次绘制的图件将分别保存到单独的sgf文件中。默认情况下,sgf图形文件的
文件名格式为~\verb+fnnn.sgf+,其中``nnn''为图件编号,起始编号为001,
每生成一个图件该编号递增。
\nameref{cmd:sgf}~命令可以控制SGF图形设备的选项,比如文件名前缀(默认为~\verb+f+~)、
起始编号(默认从~\verb+001+~开始)、保存目录、文件尺寸等。
