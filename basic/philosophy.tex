\section{SAC设计思想}
SAC的设计思想大概可以总结如下:
\begin{enumerate}
\item 每个信号\footnote{信号,或称之为trace,即单个台站单个仪器单个分量的数字记录。}
被保存到单独的SAC格式数据文件中;
\item SAC格式包含了描述数据性质的头段部分和存储信号的数据部分,参见\nameref{chap:fileformat};
\item 将单个或多个\footnote{一次性最多处理1000个任意大小的文件,记住1000这个值!}
    SAC文件从磁盘读入内存;
\item 所有操作仅对内存中的数据有效;
\item 操作完毕,将内存中的数据写入到磁盘,可以覆盖原SAC文件或写入新文件中。
\end{enumerate}
