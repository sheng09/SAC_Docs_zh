\section{去毛刺}
相关命令:\nameref{cmd:rglitches}

地震仪器偶尔会出现问题,导致连续地震数据流中出现尖锋或者数据丢失。
这些所谓的毛刺,肉眼很容易识别,但是在进行数据分析中却很容易被误认为是地震信号,
因而需要在数据分析之前将毛刺去除。数据的毛刺在模拟地震记录中很常见,现在的数字
地震记录中则很少见到。rglitches命令可以在某种程序上去除地震信号中的毛刺。

rglitches的效果可以从图~\ref{fig:deglitches}~中直观地看到。

\begin{figure}[H]
\centering
\includegraphics[width=0.75\textwidth]{rglitches}
\caption[地震波形去毛刺]{地震波形去毛刺。上图为包含glitches的地震信号,下图为去除
rglitches后的地震信号。}
\label{fig:deglitches}
\end{figure}
