\section{仪器响应}
\label{sec:instrument-response}
相关命令:\nameref{cmd:transfer}

关于仪器响应的详细介绍,请参考附录中的``\nameref{chap:resp}''一章。

SAC中~\nameref{cmd:transfer}~命令用于处理仪器响应,其基本语法是:
\begin{SACSTX}
!TRANS!FER [FROM type [!SUB!TYPE subtype]] [TO type [!SUB!TYPE subtype]]
[!FREQ!LIMITS f1 f2 f3 f4] [!PREW!HITENING ON|OFF|n]
\end{SACSTX}

transfer命令的作用是将波形数据从(from)一种仪器类型转换到(to)另一种仪器类型,
即首先将数据反卷积~\verb+from+~所给出的仪器响应,再卷积上~\verb+to+~选项给出的
仪器响应。

transfer在卷积或反卷积仪器响应时,首先需要知道仪器响应的信息。SAC中提供了四种方式
来指定仪器响应,分别是
\begin{itemize}
\item 内置仪器响应;
\item RESP文件;
\item SAC PZ文件;
\item FAP文件;
\end{itemize}

\subsection{内置仪器响应}
SAC内置了很多标准地震仪器的仪器响应,如表~\ref{table:instrument-type}~所示。
部分仪器类型还拥有子类型,如表~\ref{table:instrument-subtype}~所示。

例如,数据ABC.Z的仪器类型是LLL,通过如下命令将波形中的LLL仪器响应去除,并卷积上SRO
仪器响应:
\begin{SACCode}
SAC> r ABC.z
SAC> rmean; rtr; taper
SAC> trans from LLL to SRO      // 未使用freq和prew
\end{SACCode}

波形数据XXX.Z的仪器类型是RSTN,子类型为NYKM.Z,通过下面的命令将波形中的仪器响应
去除,并卷积上WWSP仪器响应:
\begin{SACCode}
SAC> r XXX.z
SAC> trans from RSTN subtype NYKM.Z to WWSP  // 未使用freq和prew
\end{SACCode}

除了表~\ref{table:instrument-type}~中列出的众多仪器类型之外,还有几个特别的仪器类型:
\begin{itemize}
\item none:即位移,也是SAC的默认值
\item vel:速度
\item acc:加速度
\end{itemize}

可以去除波形数据中的仪器响应,得到真实的位移场:
\begin{SACCode}
SAC> r XXX.z
SAC> trans from WWSP to NONE            // 未使用freq和prew
\end{SACCode}

当然,也可以去除仪器响应,得到真实的速度场:
\begin{SACCode}
SAC> r XXX.z
SAC> trans from WWSP to VEL
\end{SACCode}

\begin{table}[tp]
\centering
\ttfamily
\small
\caption{SAC内置仪器类型列表}
\label{table:instrument-type}
\begin{tabular}{ll}
\toprule
type	 &	说明	\\
\midrule
BBDISP   &  Blacknest specification of Broadband Displacement \\
BBVEL    &  Blacknest specification of Broadband Velocity	\\
BENBOG   &  Blacknest specification of Benioff by Bogert	\\
DSS      &  LLNL Digital Seismic System	\\
DWWSSN   &  Digital World Wide Standard Seismograph Station	\\
EKALP6   &  Blacknest specification of EKA LP6	\\
EKASP2   &  Blacknest specification of EKA SP2	\\
ELMAG    &  Electromagnetic	\\
GBALP    &  Blacknest specification of GBA LP	\\
GBASP    &  Blacknest specification of GBA SP	\\
GENERAL  &  General seismometer	\\
GSREF    &  USGS Refraction	\\
HFSLPWB  &  Blacknest specification of HFS LPWB	\\
IW       &  EYEOMG-spectral differentiation	\\
LLL      &  LLL broadband analog seismometer	\\
LLSN     &  LLSN L-4 seismometer	\\
LNN      &  Livermore NTS Network instrument	\\
LRSMLP   &  Blacknest specification of LRSM LP	\\
LRSMSP   &  Blacknest specification of LRSM SP	\\
NORESS   &  NORESS (NRSA)	\\
NORESSHF &  NORESS high frequency element	\\
OLDBB    &  Old Blacknest specification of BB	\\
OLDKIR   &  Old Blacknest specification of Kirnos	\\
PORTABLE &  Portable seismometer with PDR2	\\
PTBLLP   &  Blacknest specification of PTBL LP	\\
REDKIR   &  Blacknest specification of RED Kirnos	\\
REFTEK   &  Reftek 97-01 portable instrument	\\
RSTN     &  Regional Seismic Test Network	\\
S750     &  S750 Seismometer	\\
SANDIA   &  Sandia system 23 instrument	\\
SANDIA3  &  Sandia new system with SL-210	\\
SRO      &  Seismic Research Observatory	\\
WA       &  Wood-Anderson	\\
WABN     &  Blacknest specification of Wood-Anderson	\\
WIECH    &  Wiechert seismometer	\\
WWLPBN   &  Blacknest specification of WWSSN long period	\\
WWSP     &  WWSSN short period	\\
WWSPBN   &  Blacknest specification of WWSSN short period	\\
YKALP    &  Blacknest specification of YKA long period	\\
YKASP    &  Blacknest specification of YKA short period	\\
\bottomrule
\end{tabular}
\end{table}

\begin{table}[htb]
\centering
\ttfamily
\small
\caption{部分仪器子类型}
\label{table:instrument-subtype}
\begin{tabular}{ll}
\toprule
主类型	&	子类型	\\
\midrule
LLL       &       LV, LR, LT, MV, MR, MT, EV, ER, ET, KV, KR, KT	\\
LNN       &    	  BB|HF	                                \\
NORESS    &   	  LP|IP|SP	                            \\
RSTN      &    	  [CP|ON|NTR|NY|SD][KL|KM|KS|7S][Z|N|E]	\\
SANDIA    &   	  [N|O][T|L|B|D|N|E][V|R|T]	            \\
SRO       &       BB|SP|LPDE	                        \\
FREEPERIOD v &    ELMAG, GENERAL, IW, LLL SUBTYPE BB, REFTEK    \\
MAGNIFICATION n & ELMAG, GENERAL  \\
NZEROS n &     	  GENERAL, IW	\\
DAMPING v &    	  GENERAL, LLL SUBTYPE BB, REFTEK	\\
CORNER v &    	  LLL SUBTYPE BB, REFTEK	\\
GAIN v &		    \\
HIGHPASS v &	  REFTEK	\\
\bottomrule
\end{tabular}
\end{table}

\subsection{RESP文件}
RESP文件中包含了仪器响应的全部信息。

transfer会从SAC头段区中提取相关信息,并与RESP文件中的信息进行匹配,若二者
完全匹配方可继续执行transfer命令。通过给evalresp选项指定额外的选项和参数可以覆盖SAC文件中
对应的头段变量,这些可能的选项包括:STATION、CHANNEL、NETWORK、DATE、TIME、LOCID、FNAME。

每个选项都必须有一个合适的值,如果DATE在SAC头段中为设定且在选项中未指定,
则使用当前系统日期,TIME同理。
若NETWORK未指定,则默认使用任意台站名;若LOCID未指定,则默认使用任意LOCID。
也可以使用FNAME选项并加上RESP文件名,此时会不再检测其它信息是否匹配,强制transfer
使用指定的仪器响应文件。

如果FNAME选项未指定,EVALRESP将尝试按照一般格式在当前工作目录下寻找合适的RESP文件,
其一般格式为~``\verb+RESP.<NET>.<STA>.<LOCID>.<CHAN>+'',比如:~``\verb+RESP.IU.ANMO..BHZ+''~。

\subsubsection{EVALRESP例子}
若仪器响应文件位于当前目录,且其具有标准的文件名,可以用如下命令去除仪器响应,
transfer命令会根据SAC文件的头段区自动寻找合适的RESP文件:
\begin{SACCode}
SAC> r 2006.253.14.30.24.0000.TA.N11A..LHZ.Q.SAC
SAC> rtr
SAC> taper
SAC> trans from evalresp to none freq 0.004 0.007 0.2 0.4
\end{SACCode}

若RESP文件位于~``\verb+/tmp/Responses/RESP.TA.N11A..LHZ+''~:
\begin{SACCode}
SAC> setbb resp /tmp/Responses/RESP.TA.N11A..LHZ
SAC> r 2006.253.14.30.24.0000.TA.N11A..LHZ.Q.SAC
SAC> rtr
SAC> taper
SAC> trans from evalresp fname %resp% to none freq 0.004 0.007 0.2 0.4
\end{SACCode}

为了从文件~\verb+16.42.05.5120.TS.PAS.BHZ.SAC+~中去除仪器响应,
并添加台站COL相同周期的响应:
\begin{SACCode}
SAC> r 16.42.05.5120.TS.PAS.BHZ.SAC
SAC> rtr
SAC> taper
SAC> trans from evalresp to evalresp station COL
\end{SACCode}

为了显示IU台网COL台站BHZ通道,1992年01月02日16:42:05的仪器响应:
\begin{SACCode}
SAC> fg impulse npts 16384 delta .05 begin 0.
SAC> trans to evalresp sta COL cha BHZ net IU date 1992/2 time 16:42:05
SAC> fft
SAC> psp am
\end{SACCode}

\subsection{POLEZERO选项}
\subsubsection{SAC PZ文件}
POLEZERO选项使用SAC PZ仪器响应文件。

在批处理时,注释行信息没有用,可以写脚本重做一个无注释行的PZ文件,无注释行的PZ文件去
仪器响应的速度会比有注释行的PZ文件快很多;

\subsubsection{POLEZEROS例子}
PZ文件~\verb+SAC_PZs_XC_OR075_LHZ+~是需要从波形~\verb+OR075_LHZ.SAC+~
中去除的仪器响应零极点文件:
\begin{SACCode}
SAC> setbb pzfile "SAC_PZs_XC_OR075_LHZ"
SAC> r OR075_LHZ.SAC
SAC> rtr
SAC> taper
SAC> trans from polezero subtype %pzfile% to none freq 0.008 0.016 0.2 0.4
SAC> mul 1.0e9          // 由于是to none,这里必须乘以1.0e9
SAC> w OR075.z          // 此时数据为位移量,单位为nm
\end{SACCode}

假如在上面的例子中,没有使用~\verb+SAC_PZs_XC_OR075_LHZ+,而是使用了错误
的PZ文件~\verb+SAC_PZs_wrong+,下面的过程给出如何调用transfer命令去除不正确的
响应并加入正确的响应:
\begin{SACCode}
SAC> r OR075.z                  // 使用了错误的仪器响应文件
SAC> write OR075.zbad
SAC> setbb pzo "SAC_PZs_wrong"
SAC> setbb pzn "SAC_PZs_XC_OR075_LHZ"
SAC> trans from polezero s %pzn% to polezero s %pzo% freq 0.008 0.015 0.2 0.4
SAC> write OR075.z              // 纠正后的文件
\end{SACCode}

最后一个例子,假设我们用rdseed v5.2在当前目录产生了同一个时间的多个台站的多个波形,
并将所有的SAC PZ文件合并得到一个新的PZ文件event.pz。下面的例子将读入全部BH*波形文件,
经仪器响应校正并覆盖原文件:
\begin{SACCode}
SAC> r *BH*SAC          // 读入全部数据
SAC> rtr
SAC> taper
SAC> trans from polezero s event.pz freq 0.05 0.1 10.0 15.0
\end{SACCode}

\subsection{FAP选项}
\subsubsection{FAP文件}
SAC v101.4中重新引入了FAPfile选项,其需要FAP文件。

FAP文件是响应函数的另一种表现形式,其包含了很多记录行,每行三个字段,分别是频率(HZ)、振幅及相位。
频率不需要等间隔分段。在执行transfer时,低于第一行频率的频段将使用第一行的振幅和相位;
同理大于最后一行频率的频段将使用最后一行的振幅和相位。

EVALRESP v3.3.2可以输出为FAPfile,使用EVALRESP生成的FAPFILE而非POLEZERO的好处在于其可
以包含更丰富的仪器响应,且可以显式控制需要校正的频率段。

\subsubsection{FAPfile例子}
假设有fapfile文件~\verb+fap.n11a.lhz_0.006-0.2+,其名字表示频率段位0.006 Hz到0.2 Hz,
要从波形~\verb+2006.253.14.30.24.0000.TA.N11A..LHZ.Q.SAC+中移除该仪器响应:
\begin{SACCode}
SAC> r 2006.253.14.30.24.0000.TA.N11A..LHZ.Q.SAC
SAC> rtr
SAC> taper
SAC> trans from fap s fap.n11a.lhz_0.006-0.2 freq 0.004 0.006 0.1 0.2
SAC> mul 1.0e9
\end{SACCode}

\subsection{RESP和PZ的区别}
RESP和SAC PZ都可以用于表征仪器响应,但是这两者还是有很大区别的:

\begin{itemize}
\item RESP文件包含了仪器响应的完整信息,而PZ文件中仅包含了零极点和增益信息,
    二者的主要差异在于PZ文件中未包含FIR滤波器的信息;
\item RESP文件中可以知道输入数据是位移、速度还是加速度,而PZ文件默认输入为
    位移。因而若RESP文件中输入是速度,则PZ文件中会多一个``零''零点;若RESP
    文件中输入是加速度,则PZ文件中会多两个``零''零点;
\item RESP文件的输入单位为\verb+m+(或\verb+m/s+及其他),而PZ文件中未提供
    输入单位信息,故SAC默认其输入单位为\verb+nm+,故而使用PZ文件去除仪器响应
    时,得到的运动物理量的单位是\verb+nm+;
\end{itemize}

对于大多数情况,建议使用PZ文件,数据处理速度要快很多。
