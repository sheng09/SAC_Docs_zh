\section{去仪器响应}
\label{sec:instrument-response}
相关命令:\nameref{cmd:transfer}

关于仪器响应的详细介绍,请参考附录中的``\nameref{chap:resp}''一章。本节不会介绍
仪器响应的细节以及~\nameref{cmd:transfer}命令的语法,只介绍日常数据处理过程中
最常用的几种命令方式。

\nameref{cmd:transfer}~命令的基本语法是
``\verb+tranfser from xx to xx freq f1 f2 f3 f4+''。
即该命令的作用是将波形数据从(from)一种仪器类型转换到(to)另一种仪器类型,
即首先将数据反卷积~\verb+from+~所给出的仪器响应,再卷积上~\verb+to+~选项给出的
仪器响应,而后面的freq选项,则用于压制信号中的低频和高频成分。

SAC中的仪器类型可以简单分为如下几类:
\begin{itemize}
\item 内置仪器类型,参见\ref{table:instrument-type};
\item 特殊内置仪器类型,即none、vel和acc;
\item RESP文件;
\item PZ文件;
\item FAP文件;
\end{itemize}

下面举例说明日常数据处理中的常见用法。

数据ABC.Z的仪器类型是LLL,通过如下命令将波形中的LLL仪器响应去除,并卷积上SRO
仪器响应:
\begin{SACCode}
SAC> r ABC.z
SAC> rmean; rtr; taper
SAC> trans from LLL to SRO
\end{SACCode}

波形数据XXX.Z的仪器类型是RSTN,子类型为NYKM.Z,通过下面的命令将波形中的仪器响应
去除,并卷积上WWSP仪器响应:
\begin{SACCode}
SAC> r XXX.z
SAC> rmean; rtr; taper
SAC> trans from RSTN subtype NYKM.Z to WWSP
\end{SACCode}

从数据中去除WWSP的仪器响应,得到真实的位移场:
\begin{SACCode}
SAC> r XXX.z
SAC> rmean; rtr; taper
SAC> trans from WWSP to NONE freq 0.1 0.2 10 15
\end{SACCode}

当然,也可以去除仪器响应,得到真实的速度场:
\begin{SACCode}
SAC> r XXX.z
SAC> rmean; rtr; taper
SAC> trans from WWSP to VEL freq 0.1 0.2 10 15
\end{SACCode}

也可以使用当前台站的RESP文件来去除仪器响应:
\begin{SACCode}
SAC> r 2006.253.14.30.24.0000.TA.N11A..LHZ.Q.SAC
SAC> rmean; rtr; taper
SAC> trans from evalresp fname RESP.TA.N11A..LHZ to none freq 0.004 0.007 0.2 0.4
\end{SACCode}

PZ文件~\verb+SAC_PZs_XC_OR075_LHZ+~是需要从波形~\verb+OR075_LHZ.SAC+~
中去除的仪器响应零极点文件:
\begin{SACCode}
SAC> r OR075_LHZ.SAC
SAC> rmea; rtr; taper
SAC> trans from polezero subtype SAC_PZs_XC_OR075_LHZ to none freq 0.008 0.016 0.2 0.4
SAC> mul 1.0e9          // 由于是to none,这里必须乘以1.0e9
SAC> w OR075.z          // 此时数据为位移量,单位为nm
\end{SACCode}

假设有fapfile文件~\verb+fap.n11a.lhz_0.006-0.2+,其名字表示频率段为0.006 Hz到0.2 Hz,
要从波形~\verb+2006.253.14.30.24.0000.TA.N11A..LHZ.Q.SAC+中移除该仪器响应:
\begin{SACCode}
SAC> r 2006.253.14.30.24.0000.TA.N11A..LHZ.Q.SAC
SAC> rtr
SAC> taper
SAC> trans from fap s fap.n11a.lhz_0.006-0.2 to none freq 0.004 0.006 0.1 0.2
SAC> mul 1.0e9
\end{SACCode}
