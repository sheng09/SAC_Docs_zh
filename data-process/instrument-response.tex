\section{仪器响应}
\label{sec:instrument-response}
相关命令:\nameref{cmd:transfer}

地震仪器观测到的地面运动记录可以表示为
\[  u(t) = s(t) * g(t) * i(t) \]
其中$s(t)$代表震源项, $g(t)$代表路径效应,$i(t)$代表仪器响应,星号代表卷积。

四个量中,$u(t)$是仪器记录的结果,已知;$i(t)$在仪器设计的时候会给出各种参数,已知;
$s(t)$和$g(t)$是未知的,也是地震学研究的两个主要内容:震源和结构。
因而理解仪器响应$i(t)$的基本原理并准确地去除仪器响应是研究的关键一步。

地震仪器记录的地面运动物理量有很多,常见的有速度和加速度,不太常见的还有位移、应变、旋转。
这些物理量在被地震仪接收到之后,首先要将其转换为电信号,然后对电信号振幅进行放大以及滤波,
再将连续时间序列离散化,最终以我们常见的波形的形式表现出来,所以我们最初见到的波形
表示的是电信号的强弱,单位是counts。

数据分析经常会遇到这样的情况,需要将电信号转换成真正的地面运动物理量,此时需要去除
仪器响应;有时不同数据使用了不同类型的仪器,为了使得数据之间可以相互比较,则需要
从原始数据中去除仪器响应,并加上同一类型仪器的仪器响应。

\subsection{transfer}
transfer命令的基本语法是
\begin{SACSTX}
!TRANS!FER [FROM type [!SUB!TYPE subtype]] [TO type [!SUB!TYPE subtype]]
[!FREQ!LIMITS f1 f2 f3 f4] [!PREW!HITENING ON|OFF|n]
\end{SACSTX}

先不管FREQ和PREW这两个选项,该命令就是将波形数据从(from)一种仪器类型转换到(to)
另一种仪器类型,即反卷积~\lstinline{from}~选项给出的仪器响应,并卷积上~\lstinline{to}~
选项给出的仪器响应。

SAC内置了很多标准地震仪器的仪器响应,即type的取值如表~\ref{table:instrument-type}~所示。
部分仪器类型还拥有子类型,如表~\ref{table:instrument-subtype}~所示。

例如,数据ABC.Z的仪器类型是LLL,通过如下命令将波形中的LLL仪器响应去除,并卷积上SRO
仪器响应:
\begin{SACCode}
SAC> r ABC.z
SAC> rmean; rtr; taper
SAC> trans from LLL to SRO      // 未使用freq和prew
\end{SACCode}

波形数据XXX.Z的仪器类型是RSTN,子类型为NYKM.Z,通过下面的命令将波形中的仪器响应
去除,并卷积上WWSP仪器响应:
\begin{SACCode}
SAC> r XXX.z
SAC> trans from RSTN subtype NYKM.Z to WWSP  // 未使用freq和prew
\end{SACCode}

除了表~\ref{table:instrument-type}~中列出的众多仪器类型之外,还有几个特别的仪器类型:
\begin{itemize}
\item none:即位移,也是SAC的默认值
\item vel:速度
\item acc:加速度
\end{itemize}

可以去除波形数据中的仪器响应,得到真实的位移场:
\begin{SACCode}
SAC> r XXX.z
SAC> trans from WWSP to NONE            // 未使用freq和prew
\end{SACCode}
当然,也可以去除仪器响应,得到真实的速度场:
\begin{SACCode}
SAC> r XXX.z
SAC> trans from WWSP to VEL
\end{SACCode}

\begin{table}[tp]
\centering
\ttfamily
\small
\caption{SAC内置仪器类型列表}
\label{table:instrument-type}
\begin{tabular}{ll}
\toprule
type	 &	说明	\\
\midrule
BBDISP   &  Blacknest specification of Broadband Displacement \\
BBVEL    &  Blacknest specification of Broadband Velocity	\\
BENBOG   &  Blacknest specification of Benioff by Bogert	\\
DSS      &  LLNL Digital Seismic System	\\
DWWSSN   &  Digital World Wide Standard Seismograph Station	\\
EKALP6   &  Blacknest specification of EKA LP6	\\
EKASP2   &  Blacknest specification of EKA SP2	\\
ELMAG    &  Electromagnetic	\\
GBALP    &  Blacknest specification of GBA LP	\\
GBASP    &  Blacknest specification of GBA SP	\\
GENERAL  &  General seismometer	\\
GSREF    &  USGS Refraction	\\
HFSLPWB  &  Blacknest specification of HFS LPWB	\\
IW       &  EYEOMG-spectral differentiation	\\
LLL      &  LLL broadband analog seismometer	\\
LLSN     &  LLSN L-4 seismometer	\\
LNN      &  Livermore NTS Network instrument	\\
LRSMLP   &  Blacknest specification of LRSM LP	\\
LRSMSP   &  Blacknest specification of LRSM SP	\\
NORESS   &  NORESS (NRSA)	\\
NORESSHF &  NORESS high frequency element	\\
OLDBB    &  Old Blacknest specification of BB	\\
OLDKIR   &  Old Blacknest specification of Kirnos	\\
PORTABLE &  Portable seismometer with PDR2	\\
PTBLLP   &  Blacknest specification of PTBL LP	\\
REDKIR   &  Blacknest specification of RED Kirnos	\\
REFTEK   &  Reftek 97-01 portable instrument	\\
RSTN     &  Regional Seismic Test Network	\\
S750     &  S750 Seismometer	\\
SANDIA   &  Sandia system 23 instrument	\\
SANDIA3  &  Sandia new system with SL-210	\\
SRO      &  Seismic Research Observatory	\\
WA       &  Wood-Anderson	\\
WABN     &  Blacknest specification of Wood-Anderson	\\
WIECH    &  Wiechert seismometer	\\
WWLPBN   &  Blacknest specification of WWSSN long period	\\
WWSP     &  WWSSN short period	\\
WWSPBN   &  Blacknest specification of WWSSN short period	\\
YKALP    &  Blacknest specification of YKA long period	\\
YKASP    &  Blacknest specification of YKA short period	\\
\bottomrule
\end{tabular}
\end{table}

\begin{table}[htb]
\centering
\ttfamily
\small
\caption{部分仪器子类型}
\label{table:instrument-subtype}
\begin{tabular}{ll}
\toprule
主类型	&	子类型	\\
\midrule
LLL       &       LV, LR, LT, MV, MR, MT, EV, ER, ET, KV, KR, KT	\\
LNN       &    	  BB|HF	                                \\
NORESS    &   	  LP|IP|SP	                            \\
RSTN      &    	  [CP|ON|NTR|NY|SD][KL|KM|KS|7S][Z|N|E]	\\
SANDIA    &   	  [N|O][T|L|B|D|N|E][V|R|T]	            \\
SRO       &       BB|SP|LPDE	                        \\
FREEPERIOD v &    ELMAG, GENERAL, IW, LLL SUBTYPE BB, REFTEK    \\
MAGNIFICATION n & ELMAG, GENERAL  \\
NZEROS n &     	  GENERAL, IW	\\
DAMPING v &    	  GENERAL, LLL SUBTYPE BB, REFTEK	\\
CORNER v &    	  LLL SUBTYPE BB, REFTEK	\\
GAIN v &		    \\
HIGHPASS v &	  REFTEK	\\
\bottomrule
\end{tabular}
\end{table}

随着地震仪器的不断发展,地震仪器的类型也越来越多,SAC不可能包含已知的全部仪器
类型,因而需要更灵活的方式来指定仪器响应。SAC提供了三种方式:EVALRESP、POLEZERO
和FAP。

\subsection{EVALRESP选项}
\subsubsection{RESP文件}
该选项需要RESP文件。RESP文件包含了仪器的完整信息,其可以由rdseed程序直接从SEED数据
中解压得到。想要了解RESP文件的具体细节,可以参考SEED格式的说明文档:
\url{http://www.fdsn.org/seed_manual/SEEDManual_V2.4.pdf}。

RESP文件中包含了仪器响应的零极点信息及其所对应的台网名、台站名、通道名、日期和时间
等信息。transfer会从SAC头段区中提取相关信息,并与RESP文件中的信息进行匹配,若二者
完全匹配方可继续执行transfer命令。通过给evalresp选项指定额外的选项和参数可以覆盖SAC文件中
对应的头段变量,这些可能的选项包括:STATION、CHANNEL、NETWORK、DATE、TIME、LOCID、FNAME。

每个选项都必须有一个合适的值,如果DATE在SAC头段中为设定且在选项中未指定,
则使用当前系统日期,TIME同理。
若NETWORK未指定,则默认使用任意台站名;若LOCID未指定,则默认使用任意LOCID。
也可以使用FNAME选项并加上RESP文件名,此时会不再检测其它信息是否匹配,强制transfer
使用指定的仪器响应文件。

如果FNAME选项未指定,EVALRESP将尝试按照一般格式在当前工作目录下寻找合适的RESP文件,
其一般格式为~``\lstinline{RESP.<NET>.<STA>.<LOCID>.<CHAN>}'',比如:~``\lstinline{RESP.IU.ANMO..BHZ}''~。

evalresp本质上是独立的程序,SAC将其内嵌到SAC的代码中。由于evalresp默认使用$m$作为基本单位,
而SAC以$nm$作为基本单位,在使用evalresp选项对波形数据去仪器响应得到位移场、速度场或
加速度场时,SAC会自动完成$m$到$nm$的转换,对数据乘以$10^9$的因子以符合SAC的标准。

\subsubsection{EVALRESP例子}
若仪器响应文件位于当前目录,且其具有标准的文件名,可以用如下命令去除仪器响应,
transfer命令会根据SAC文件的头段区自动寻找合适的RESP文件:
\begin{SACCode}
SAC> r 2006.253.14.30.24.0000.TA.N11A..LHZ.Q.SAC
SAC> rtr
SAC> taper
SAC> trans from evalresp to none freq 0.004 0.007 0.2 0.4
\end{SACCode}

若RESP文件位于~``\lstinline{/tmp/Responses/RESP.TA.N11A..LHZ}''~:
\begin{SACCode}
SAC> setbb resp /tmp/Responses/RESP.TA.N11A..LHZ
SAC> r 2006.253.14.30.24.0000.TA.N11A..LHZ.Q.SAC
SAC> rtr
SAC> taper
SAC> trans from evalresp fname %resp% to none freq 0.004 0.007 0.2 0.4
\end{SACCode}

为了从文件~\lstinline{16.42.05.5120.TS.PAS.BHZ.SAC}~中去除仪器响应,
并添加台站COL相同周期的响应:
\begin{SACCode}
SAC> r 16.42.05.5120.TS.PAS.BHZ.SAC
SAC> rtr
SAC> taper
SAC> trans from evalresp to evalresp station COL
\end{SACCode}

为了显示IU台网COL台站BHZ通道,1992年01月02日16:42:05的仪器响应:
\begin{SACCode}
SAC> fg impulse npts 16384 delta .05 begin 0.
SAC> trans to evalresp sta COL cha BHZ net IU date 1992/2 time 16:42:05
SAC> fft
SAC> psp am
\end{SACCode}

\subsection{POLEZERO选项}
\subsubsection{SAC PZ文件}
POLEZERO选项需要使用另一种仪器响应文件,称之为SAC PZ文件。

RESP文件中包含了仪器响应的所有细节,因而相对来说较为繁琐。SAC从RESP文件中提取出
仪器响应中的相关信息,重新定义了新的PZ响应文件。相对于RESP文件来说,PZ文件中仅
包含仪器响应的零极点和增益信息,在去仪器响应时速度更快。

地震仪器的响应函数(即线性系统脉冲响应)一般用Laplace域的零极点来表示。响应函数
$H(s)$由$nz$个零点和$np$个极点构成,如下式所示:
\[ H(s) = \frac{(s-z_1)(s-z_2)\ldots(s-z_{nz})}{(s-p_1)(s-p_2)\ldots(p-z_{np})}\]
其中,Laplace变量$s=2\pi f i$,$f$为频率,单位$Hz$。

PZ文件中宏包含零点、极点、常数值以及注释行,其中常数是比例因子。常数行的缺省值为1.0。

文件中包含关键字``POLES''的行给出了极点个数(np),接下来的np行列出仪器响应的极点,
每行两个浮点数,分别代表实部和虚部。

以关键字``ZEROS''起始的行给出零点个数(nz)。一般地,零极点文件中可能包含一个或多个
~\lstinline{(0.0,0.0)}~这样的零点,在SAC中这样的零点不出现在PZ文件中,所以接下来
列出的零点数可能会少于nz个。

PZ文件中还包含格式化的注释行信息,以帮助用户组织和理解PZ文件中的内容。目前PZ零极点
文件一般用rdseed从SEED中解压得到或者从SAC PZ网页服务\footnote{\url{http://www.iris.edu/ws/sacpz}}
中获取。若内存中的波形数据头段区KSTNM、KCMNM、KZDATE、KZTIME与零极点文件中的注释
信息相匹配,则transfer命令会正常运行。一个SAC PZ文件中,可能包含多个仪器的零极点文件的
信息,其跨越了多个时间段或包含多个台站多个通道,transfer命令会对内存中与之匹配的
波形数据进行处理。

下面给出rdseed v5.2产生的一个零极点文件的例子,早期版本的rdseed的输出可能与此略微不同:

\begin{SACCode}
    * **********************************
    * NETWORK   (KNETWK): II
    * STATION    (KSTNM): PFO
    * LOCATION   (KHOLE): 00
    * CHANNEL   (KCMPNM): BHZ
    * CREATED           : 2011-08-11T00:24:07
    * START             : 2010-07-30T18:50:00
    * END               : 2599-12-31T23:59:59
    * DESCRIPTION       : Pinon Flat, California, USA
    * LATITUDE          : 33.610700
    * LONGITUDE         : -116.455500
    * ELEVATION         : 1280.0
    * DEPTH             : 5.3
    * DIP               : 0.0
    * AZIMUTH           : 0.0
    * SAMPLE RATE       : 20.0
    * INPUT UNIT        : M
    * OUTPUT UNIT       : COUNTS
    * INSTTYPE          : Streckeisen STS-1 Seismometer with Metrozet E300
    * INSTGAIN          : 3.314400e+03 (M/S)
    * COMMENT           : S/N #119005
    * SENSITIVITY       : 5.247780e+09 (M/S)
    * A0                : 7.273290e+01
    * **********************************
    ZEROS	6
    -7.853982e+01	+0.000000e+00
    -1.525042e-01	+0.000000e+00
    -1.525042e-01	+0.000000e+00
    POLES	6
    -1.207063e-02	+1.224561e-02
    -1.207063e-02	-1.224561e-02
    -1.522510e-01	+9.643684e-03
    -1.522510e-01	-9.643684e-03
    -4.832398e+01	+5.817080e+01
    -4.832398e+01	-5.817080e+01
    CONSTANT	3.816863e+11
\end{SACCode}

该仪器响应表明,响应函数含六个极点和六个零点(只列出三个非0零点)。

在某些平台上,rdseed v5.1会产生错误的end时间,通常比start时间早一点,而导致
transfer无法正常工作,手动将end时间修改为迟于created时间可以解决这个问题。
SAC为了解决这个问题,可以指定仪器类型为POLEZERO,子类型为PZ文件名。此时SAC
会强制使用该PZ文件作为响应文件,而不去检测头段是否匹配。在指定子类型为PZ文件名
的前提下,PZ文件中的注释行便不再被transfer命令所需要,因而有或者没有对于去仪器
响应而言没有影响。

在处理大量数据时做了一下测试,一种情况是直接使用rdseed解压出的包含注释的PZ文件去仪器响应,
另一种是用脚本重做一个无注释的PZ文件,再用无注释的PZ文件去仪器响应。测试结果表明,
后者在速度上相对于前者快了非常多。

需要注意的是,如果零极点文件由rdseed从SEED文件中提取得到,则零极点文件的输入为位移,
其单位与SEED格式相同,一般为$m$。不像evalresp,SAC不知道零极点文件来自何处,也无法
知道其输入单位是$m$,因此在使用PZ文件去除仪器响应到位移量、速度量或加速度量时,
应手动乘以比例因子$10^9$以保证数据的正确性。

\subsubsection{POLEZEROS例子}
PZ文件~\lstinline{SAC_PZs_XC_OR075_LHZ}~是需要从波形~\lstinline{OR075_LHZ.SAC}~
中去除的仪器响应零极点文件:
\begin{SACCode}
SAC> setbb pzfile "SAC_PZs_XC_OR075_LHZ"
SAC> r OR075_LHZ.SAC
SAC> rtr
SAC> taper
SAC> trans from polezero subtype %pzfile% to none freq 0.008 0.016 0.2 0.4
SAC> mul 1.0e9          // 由于是to none,这里必须乘以1.0e9
SAC> w OR075.z          // 此时数据为位移量,单位为nm
\end{SACCode}

假如在上面的例子中,没有使用~\lstinline{SAC_PZs_XC_OR075_LHZ},而是使用了错误
的PZ文件~\lstinline{SAC_PZs_wrong},下面的过程给出如何调用transfer命令去除不正确的
响应并加入正确的响应:
\begin{SACCode}
SAC> r OR075.z                  // 使用了错误的仪器响应文件
SAC> write OR075.zbad
SAC> setbb pzo "SAC_PZs_wrong"
SAC> setbb pzn "SAC_PZs_XC_OR075_LHZ"
SAC> trans from polezero s %pzn% to polezero s %pzo% freq 0.008 0.015 0.2 0.4
SAC> write OR075.z              // 纠正后的文件
\end{SACCode}

最后一个例子,假设我们用rdseed v5.2在当前目录产生了同一个时间的多个台站的多个波形,
并将所有的SAC PZ文件合并得到一个新的PZ文件event.pz。下面的例子将读入全部BH*波形文件,
经仪器响应校正并覆盖原文件:
\begin{SACCode}
SAC> r *BH*SAC          // 读入全部数据
SAC> rtr
SAC> taper
SAC> trans from polezero s event.pz freq 0.05 0.1 10.0 15.0
\end{SACCode}

\subsection{FAP选项}
\subsubsection{FAP文件}
SAC v101.4中重新引入了FAPfile选项,其需要FAP文件。

FAP文件是响应函数的另一种表现形式,其包含了很多记录行,每行三个字段,分别是频率(HZ)、振幅及相位。
频率不需要等间隔分段。在执行transfer时,低于第一行频率的频段将使用第一行的振幅和相位;
同理大于最后一行频率的频段将使用最后一行的振幅和相位。

EVALRESP v3.3.2可以输出为FAPfile,使用EVALRESP生成的FAPFILE而非POLEZERO的好处在于其可
以包含更丰富的仪器响应,且可以显式控制需要校正的频率段。

\subsubsection{FAPfile例子}
假设有fapfile文件~\lstinline{fap.n11a.lhz_0.006-0.2},其名字表示频率段位0.006 Hz到0.2 Hz,
要从波形~\lstinline{2006.253.14.30.24.0000.TA.N11A..LHZ.Q.SAC}中移除该仪器响应:
\begin{SACCode}
SAC> r 2006.253.14.30.24.0000.TA.N11A..LHZ.Q.SAC
SAC> rtr
SAC> taper
SAC> trans from fap s fap.n11a.lhz_0.006-0.2 freq 0.004 0.006 0.1 0.2
SAC> mul 1.0e9
\end{SACCode}

\subsection{其它}

关于仪器响应的更多细节可以查看SEED格式手册以及参考如下几篇博文:
\begin{itemize}
\item \href{http://seisman.info/instrumental-response-in-seismology.html}{地震学的仪器响应}
\item \href{http://seisman.info/physical-details-of-instrumental-response.html}{仪器响应的物理细节}
\item \href{http://seisman.info/simple-analysis-of-resp.html}{仪器响应文件RESP}
\item \href{http://seisman.info/simple-analysis-of-sac-pz.html}{仪器响应文件SAC PZ}
\item \href{http://seisman.info/difference-between-resp-and-pz-while-deconvolution.html}{用RESP和PZ去除仪器响应的差别}
\item \href{http://seisman.info/deep-analysis-of-response.html}{仪器响应实例分析}
\end{itemize}
