\section{数据截窗}

相关命令:\nameref{cmd:cut}、\nameref{cmd:cutim}

数据申请时一般会选择尽可能长的时间窗,而实际进行数据处理和分析时可能只需要其中
的一小段时间窗,这就需要对数据时间窗进行截取。

SAC中有两个命令可以用于数据截窗,分别是cut和cutim,二者的语法基本是相同的,
需要注意的是cut是``参数设定类''命令,cutim是``操作执行类''命令。

除此之外,在进行数据测量时,也经常需要定义时间窗,如rms、mtw、xlim等。

\subsection{PDW}
\label{subsec:pdw}
SAC中用pdw来定义要读取的时间窗。

pdw即partial data window,其包含了自变量的一个开始值和一个结束值,定义了要读取的
数据窗。其一般格式为~\verb+ref offset ref offset+。
其中ref为参考时刻标识,可以取~\verb+Z|B|E|O|A|F|N|Tn+,而offset为相对于参考
时刻的时间偏移量。

若开始或结束offset省略则认为其值为0。若开始ref省略则认为其为Z;若结束ref省略则
认为其值与开始ref相同。

参考时刻标识ref可以取如下值:
\begin{itemize}
\item B: 磁盘文件起始值
\item E: 磁盘文件结束值
\item O: 事件开始时间
\item A: 初动到时
\item F: 信号结束时间
\item Tn: 用户自定义时间标记,n = 0,1...9
\item Z: 参考时刻
\item N: 将offset解释为数据点数而非时间偏移量,其仅可以用于结束值
\end{itemize}

下面的例子中展示了一些常见的pdw:
\begin{SACCode}
 B E        // 文件开始到文件结束,即与cut off相同
 B 0 30     // 文件开始的30秒
 A -10 30   // 初动前10秒到初动后30秒
 B N 2048   // 文件最初的2048个点
 30.2 48    // 相对磁盘文件0点的30.2到48秒
\end{SACCode}
