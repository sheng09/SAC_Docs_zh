\section{合并数据}
相关命令: \nameref{cmd:merge}

有些时候,从SEED格式中解压出来的连续波形数据会被切割成多个等长或不等长的数据段,
可能是因为仪器在某些时刻存在问题导致连续数据出现间断,也可能是出于其它考虑将数据进行切割。
用户需要将这些数据段合并成单个包含连续波形数据的文件。

假定有三段连续/非连续的数据需要合并在一起,需要首先读入第一段数据,再merge其它段
数据,最后使用w over将合并后的数据写入到第一个文件中,此时其余段数据segment2.SAC
和segment3.SAC就可以删除了。

\begin{SACCode}
SAC> r segment1.SAC
SAC> merge segment2.SAC
SAC> merge segment3.SAC
SAC> w over
\end{SACCode}

需要注意的是,在数据合并的过程中,可能会出现很多意外情况。

如果第一段数据的结束时刻早于第二段数据的开始时刻,即两段数据在时间上存在间断(gap),
merge命令提供了选项可以控制在合并数据时如何对数据间断进行操作,是直接补零(zero)还是
做线性插值(interp)。

如果第一段数据的结束时间晚于第二段数据的开始时间,则两段数据在时间上存在重叠(overlap)。
对于这种情况,可以比较(compare)两段数据在重叠的时间段内的波形是否相同,若相同则正常
合并,否则直接退出;也可对重叠的时间段内的波形直接做平均(average)。
