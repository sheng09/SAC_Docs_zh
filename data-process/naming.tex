\section{数据的命名}
从压缩格式中解压得到的SAC数据,其命名方式不够友好,比如用rdseed解压得到的数据,
文件名的形式如下:
\begin{lstlisting}[style=Bash]
2012.055.12.34.56.7777.YW.MAIO.01.BHE.Q.SAC
2012.055.12.34.50.6666.YW.MAIO.01.BHN.Q.SAC
2012.055.12.34.54.5555.YW.MAIO.01.BHZ.Q.SAC
\end{lstlisting}
三个文件代表了YW台网MAIO台站的宽频地震仪记录的三分量波形数据。这样的长文件名在数据处理
时显得很麻烦,一般都会先进行适当的简化。

在某些情况下,我们会将同一个事件的所有数据放在同一个文件夹下,并将文件名以事件的
发生日期/时间来命名。那么,SAC文件名中的时间信息就可以被简化:
\begin{lstlisting}[style=Bash]
YW.MAIO.BHE
YW.MAIO.BHN
YW.MAIO.BHZ
\end{lstlisting}

有时候,我们会将不同事件在同一个台站的波形数据放在同一个文件夹下,并将文件名以台站
名来命名,此时数据文件名中可能需要保留事件的日期信息:
\begin{lstlisting}[style=Bash]
MAIO.20120224.BHE
MAIO.20120224.BHN
MAIO.20120224.BHZ
\end{lstlisting}

鉴于SAC命令的语法,在数据命名时最好将分量名放在最后,而将台站名放在最前面。
这样,在使用SAC的通配符读取特定事件的所有台站的垂直分量波形数据时,可以:
\begin{SACCode}
SAC> r *.20120224.BHZ
\end{SACCode}
或者读取所有事件在同一台站的波形记录:
\begin{SACCode}
SAC> r MAIO.*.BHZ
\end{SACCode}
