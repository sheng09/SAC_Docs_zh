\section{事件信息}
相关头段:evla, evlo, evdp, mag, o, nzyear, nzjday, nzhour, nzmin, nzsec, nzmsec

一般来说,从SEED连续波形中解压得到的SAC数据中是没有事件信息的。这需要用户搜索
地震目录,获取事件的发震时刻、经度、纬度、深度和震级信息,并将这些信息写入到
SAC文件的头段中。

需要用到的命令是\nameref{cmd:chnhdr}(简写为ch)。其语法相对来说比较简单,比如
想要修改事件的位置和深度,可以操作如下:
\begin{SACCode}
SAC> r cdv.?
SAC> ch evla 37.52 evlo -121.68 evdp 5.95   // 修改三个头段变量
SAC> ch mag 5.0                             // 修改一个头段变量
SAC> wh                                     // 将修改后的头段写入文件
\end{SACCode}

下面试图加入发震时刻的信息。前面已经说过,我们一般将SAC文件的参考时刻设置为发震
时刻,这样的处理可以带来一堆好处。ch提供了专门的选项用于处理发震信息。

假设发震时刻为1987年06月22日11时10分10.363秒:
\begin{SACCode}
SAC> r ./cdv.?
SAC> ch o gmt 1987 173 11 10 10 363
SAC> lh kzdate kztime o
  
  FILE: ./cdv.e - 1
 -------------

     kzdate = JUN 22 (173), 1987
     kztime = 11:09:56.363
          o = 1.400000e+01
  
SAC> ch allt -14
SAC> lh kzdate kztime o
  
  FILE: ./cdv.e - 1
 -------------

     kzdate = JUN 22 (173), 1987
     kztime = 11:10:10.363
          o = 0.000000e+00
SAC> wh
\end{SACCode}

首先需要计算发震日期对应一年中的第几天,本例中为第173天,然后利用\\
\lstinline{ch o gmt yyyy ddd hh mm sss xxx}的语法将发震时刻赋值给头段变量o。
此时文件的参考时刻为1987-06-22T11:09:56.363,修改之后的o对应的时刻为发震时刻,
其相对于参考时刻的时间差为14s。

\lstinline{ch allt xx.xx}的功能是将参考时刻的值加上xx.xx秒,同时将所有的相对时间
减去xx.xx秒,此时参考时刻即为发震时刻,而o值为0。

在需要批量处理数据的时候,不可能通过lh o 的方式查看o的值,然后使用到allt中。在
SAC宏一节中提供了适于批量处理的方式:
\begin{SACCode}
SAC> ch o gmt 1987 173 11 10 10 363 
SAC> ch allt (0 - &1,o&) iztype IO      
\end{SACCode}

