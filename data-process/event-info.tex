\section{事件信息}
\label{sec:event-info}
相关头段:evla, evlo, evdp, mag, o, nzyear, nzjday, nzhour, nzmin, nzsec, nzmsec

一般来说,从SEED连续波形中解压得到的SAC数据中是没有事件信息的。这需要用户搜索
地震目录,获取事件的发震时刻、经度、纬度、深度和震级信息,并将这些信息写入到
SAC文件的头段中。

可以使用~\nameref{cmd:chnhdr}~修改头段变量,再用~\nameref{cmd:writehdr}~将修改后
的头段变量写回SAC文件中。比如
想要修改事件的位置和深度,可以操作如下:
\begin{SACCode}
SAC> r cdv.?
SAC> ch evla 37.52 evlo -121.68 evdp 5.95   // 修改三个头段变量
SAC> ch mag 5.0                             // 修改一个头段变量
SAC> wh                                     // 将修改后的头段写入文件
\end{SACCode}

这里需要注意~``\verb+wh+''~与~``\verb+w over+''~的区别。二者都是将内存中的
数据写回到原文件中。前者只覆盖文件的头段区,后者则覆盖文件的头段区和数据区。
若SAC数据从磁盘读入内存之后,未对波形数据做任何修改,仅修改了头段区,此时应
使用~``\verb+wh+''~命令将修改后的头段区写回文件,相对于~``\verb+w over+''来说速度要快很多。

向SAC头段中加入发震时刻的信息,要稍稍麻烦一点。前面已经说过,一般将SAC文件的参考时刻
设置为发震时刻,这样的处理可以带来一堆好处。ch提供了专门的选项用于处理发震信息。

假设发震时刻为1987年06月22日11时10分10.363秒:
\label{code:origin-time}
\begin{SACCode}
SAC> r ./cdv.?
SAC> ch o gmt 1987 173 11 10 10 363
SAC> lh kzdate kztime o
  
  FILE: ./cdv.e - 1
 -------------

     kzdate = JUN 22 (173), 1987
     kztime = 11:09:56.363
          o = 1.400000e+01
  
SAC> ch allt -14
SAC> ch iztype IO
SAC> lh kzdate kztime o
  
  FILE: ./cdv.e - 1
 -------------

     kzdate = JUN 22 (173), 1987
     kztime = 11:10:10.363
          o = 0.000000e+00
SAC> wh
\end{SACCode}

在上面的例子中,首先从地震目录中获取了地震的发震时刻,然后计算发震日期对应一年中的第几天,
本例中为第173天,再利用``\verb+ch o gmt yyyy ddd hh mm sss xxx+''
的语法将发震时刻赋值给头段变量o,SAC会自动将发震时刻转换为相对于参考时刻的相对时间。
由于此时SAC文件的参考时刻为``1987-06-22T11:09:56.363'',而o值对应的时刻为发震时刻
``1987-06-22T11:10:10.363'',所以头段变量o的值为发震时刻相对于参考时刻的时间差,即14s。

将发震时刻写入头段之后,还需要将参考时刻修改为发震时刻,与此同时还要修改所有的相对时间。
``\verb+ch allt xx.xx+''的功能是将所有已定义的相对时间加上xx.xx秒,
同时从参考时刻中减去xx.xx秒,此时参考时刻即为发震时刻,而o值为0。

在需要批量处理数据的时候,不可能通过``lh o''的方式查看o的值,然后使用到allt中。在
~\nameref{chap:sac-programming}~一章中提供了适于批量处理的方式:
\begin{SACCode}
SAC> ch o gmt 1987 173 11 10 10 363 
SAC> ch allt (0 - &1,o&) iztype IO
\end{SACCode}

\begin{Tips}
上例中的~``\verb+(0 - &1,o&)+''~必须原样输入!
    
由于101.6版本重写了SAC宏的语法分析器,所以很多用法都发生了改变。你可能会发现,在
101.6及其以后的版本中,\verb+(0-&1,o&)+、\verb+(-&1,o&)+、\verb+(-&1,o)+都
可以正常使用,但是上例中的版本是唯一一个在所有SAC版本中都正确的,为了命令的通用性,
只要记住这个就可以了。
\end{Tips}
