\section{Bash中调用SAC}
\label{sec:sac-bash}

\subsection{简介}
SAC宏的功能相对比较单一,难以满足日常数据处理的需求,可以在Bash脚本中直接调用SAC,
这样可以利用Bash脚本的更多特性。

下面的例子展示了如何在Bash脚本中调用SAC:
\inputminted{bash}{./call-in-script/simple-script.sh}

SAC在启动是默认会显示版本信息,当用脚本多次调用SAC时,版本信息也会显示多次,可以
通过设置环境变量~``\lstinline{export SAC_DISPLAY_COPYRIGHT=0}''的方式隐藏版本信息。

脚本中从~``\lstinline{sac << EOF}''~开始到~``\lstinline{EOF}''~的全部内容,都会被
Bash传递给SAC,SAC会逐一解释并执行每行命令。

\subsection{头段变量和黑板变量}
想要在Bash脚本中引用头段变量,需要借助于SAC宏的语法。
\inputminted{bash}{./call-in-script/variables.sh}

\subsection{内联函数}
bash可以完成基本的数学运算,但是所有的运算只支持整型数据,浮点型运算或者其它更
高级的数学运算需要借助bc或者awk来完成。Bash中的变量以~``\lstinline{$}''~作为标识符,
Bash会首先做变量替换再将替换后的命令传递给SAC。
\inputminted{bash}{./call-in-script/arithmetic-functions.sh}

本例中的变量~``\lstinline{$var1}''~和~``\lstinline{$var2}''~会首先被SAC解释成
为1和2,因而SAC实际接收到的命令是~``\lstinline{bp c 1 2}''。

借助于awk、sed等工具,也可以实现部分字符串处理函数:
\inputminted{bash}{./call-in-script/string-functions.sh}

\subsection{条件判断和循环控制}
Bash具有更灵活的条件判断和循环控制功能,但由于Bash自身的限制,这些特性仅能
在SAC外部使用,因而下例中需要多次调用SAC,在某些情况下会相当耗时。
\inputminted{bash}{./call-in-script/do-loops.sh}
